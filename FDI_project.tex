% -------------------------------------------------------------
%		Preamble
%--------------------------------------------------------------
% !TeX spellcheck = uk_UK
%Document Settings
\documentclass[a4paper,11pt]{scrartcl}
\usepackage[T1]{fontenc}
\usepackage{setspace}
\setlength\parindent{24pt}
\onehalfspacing
\DeclareOldFontCommand{\bf}{\normalfont\bfseries}{\mathbf} % Includes an old command that is part of some package in here (just to make sure everything runs)

% Page numbers
\newcommand{\sectionnumbering}[1]{%
  \setcounter{section}{0}%
   \renewcommand{\thesection}{\csname #1\endcsname{section}}}

% Language settings
\usepackage[utf8]{inputenc}
\usepackage[english]{babel}

% Maths settings
\usepackage{amsmath}
\newcommand\numberthis{\addtocounter{equation}{1}\tag{\theequation}}
\usepackage{amssymb}
\usepackage{siunitx}


% Graphics packages
\usepackage{graphicx}
\graphicspath{{Graphics/}}
\usepackage{adjustbox}
\usepackage{subcaption}

% Table packages
\usepackage{pdflscape} %To create landscape environments
\usepackage{booktabs,caption}
\usepackage{threeparttable} %For nice tables
\usepackage{csvsimple} %To import excel tables 
\usepackage{import}
\usepackage{longtable}

% Bilbiography settings
\usepackage{natbib}
\bibliographystyle{apalike}

%Commenting over multiple rows
\newcommand{\comment}[1]{}



%-------------------------------------------------------------
%		Title Page
%--------------------------------------------------------------
\begin{document}

	\begin{titlepage}
		\newcommand{\HRule}{\rule{\linewidth}{0.5mm}}
		
%% Logo
	\vfill\vfill
	\includegraphics[height=1.5cm]{UoN_Logo}\\[1cm] 

	\vfill\vfill
	\center			
%% Heading
 
	\textsc{\Large Applied Microeconometrics}\\[0.5cm] 	
	\textsc{\large Group Project A}\\[0.5cm] 
	
%% Title
	\HRule\\[0.4cm]
	{\huge\bfseries The Effect of FDI on Firm Productivity - A Propensity Score Estimation Approach}\\[0.4cm] 
	\HRule\\[0.4cm]
	
%% Date
	{\large\ Spring Term 2020} 	
	\vfill\vfill\vfill\vfill	
	
%% Author(s) and Supervisor
\begin{flushleft}
			\large
			\textit{Supervisor}\\
			Professor Sourafel \textsc{Girma} 
			\vfill\vfill 
			\textit{Authors}\\
			Emilie \textsc{Bechtold} (20214031)\\
			Nelly  \textsc{Lehn} (20214338)\\
			Yonesse \textsc{Paris} (20116536)\\
			Georg  \textsc{Schneider} (20214032)\\
			Thea  \textsc{Zoellner} (20216019)
		\end{flushleft}
	\vfill 
	
\end{titlepage}


% -------------------------------------------------------------
%		Contents
%--------------------------------------------------------------
\pagenumbering{roman}
\sectionnumbering{Roman}
\tableofcontents

\newpage

\listoftables
\listoffigures
\newpage

%-------------------------------------------------------------
% Main Body
%-------------------------------------------------------------
\pagenumbering{arabic}
\sectionnumbering{arabic}

\section{Introduction}

The understanding of potential effects of Foreign Direct Investment (FDI) on a firm's productivity is of major concern to policy makers. FDI is commonly associated with higher firm productivity \citep{girma2007}. Despite many efforts of researchers to identify the causal mechanisms underlying this correlation, it remains difficult to pin down the size and direction of the relationship. Most argue that foreign investment positively impacts firm productivity. However, it is also possible that foreign investors choose %pick
more productive firms \citep{arnold2009}. %reversing the direction of causality.

Identification of%In order to identify 
the causal effect of FDI on a firm’s performance, %and 
in particular %on 
its total-factor productivity (TFP), %one would need to know 
requires the counterfactual outcome. Although it is inherently unobservable, different methods can be used to take into account the biases stemming from this missing data problem, e.g. randomization, Difference-in-Differences (DiD), as well as %or
Instrumental Variable and Propensity Score methods \citep{karpaty2007}. A common method in the economic literature combines DiD with Propensity Score-based estimators. The latter %propensity score matching estimator 
is used in order to compare treated to those untreated firms which are similar in their likelihood of receiving treatment, given a set of observable pre-acquisition characteristics %(e.g. size, industry characteristics). 
DiD estimation on the other hand accounts for unobservable firm characteristics that are constant over time. Estimations combining both methods provide a robust Average Treatment Effect (ATE). 

This methodology is used by \citet{arnold2009, karpaty2007, girma2007} and \citet{schiffbauer2017}. \citet{arnold2009}  find a positive and persistent effect of FDI on firm productivity, estimating a 13.5\% increase in productivity of treated firms after three years.\citet{karpaty2007} finds a positive effect of foreign acquisitions on productivity in Swedish manufacturing ranging between 7 and 8 percent for the DiD estimators. However, it can take up to five years for productivity differences to occur. 

%Using this methodology, \citet{arnold2009}  find a positive and persistent effect of FDI on firm productivity. Their analysis suggests a 13.5\% increase in productivity of treated firms after three years. The same method is used by \citet{karpaty2007, girma2007} and \citet{schiffbauer2017}. \citet{karpaty2007} finds a positive effect of foreign acquisitions on productivity in Swedish manufacturing ranging between 7 and 8 percent for the DiD estimations. 
%Moreover, it can take up to five years for productivity differences to occur. 
\citet{girma2007} use plant-level data for the UK for the electronic and food industries and find substantial heterogeneity across industries, especially with respect to the onset of positive effects on TFP growth. Their results suggest that any positive impact of foreign acquisitions is mainly explained by changes in technical efficiency rather than scale effects. \citet{koch2019} combine DiD with Inverse Propensity Score Weighting (IPW). They use Spanish firm level data %from 1998 to 2013, 
providing evidence of an increase in output of ten percent which is explained almost entirely through skill upgrading caused by foreign acquisition. 

Using various Propensity Score estimators, we investigate the effect of FDI on TFP for a sample of 11,323 firms. In line with previous research, we %are able to 
identify a statistically and economically significant effect of FDI on firm productivity, with an ATE of about 13 %to 15
 percent of a standard deviation. This result is robust to various model specifications, although there seems to be some heterogeneity of the effect across different levels of technology intensity. We also examine the effects of the specific types of FDI, but find no evidence of differences in their impact on firm productivity.

The remainder of this paper is organized as follows: The data and empirical specification are presented in sections 2 and 3, respectively. The results and robustness checks are shown in section 4. Section 5 concludes.

\section{Data and Descriptive Analysis}
Our analysis is based on observational firm-level data ranging from 2015 to 2017. The dataset comprises 11,323 firms, of which 4,460 received FDI in 2016. FDI can be divided into three subcategories. Table \ref{tab:freq} shows the frequencies of each type of FDI in our sample. Among the recipients of FDI, most firms (1,965) received domestic market seeking FDI. 1,555 firms received technology intensive FDI and the remaining 640 firms received exports oriented FDI. The outcome variable TFP was measured in 2017 %after the treatment. 
We standardize TFP %the outcome variable 
to a mean of zero and a standard deviation of one, making the interpretation more intuitive. 

%--------------------TABLE 1
\begin{table}[h!]
	\centering
	\caption{Frequency of FDI Types} 
	\begin{threeparttable}
\begin{tabular}{lcc} 
	\toprule \toprule 
	FDI type&Abs. Freq.&Rel. Freq. \\[1ex]
	\midrule
	No FDI							& 6,863		& 61\% \\
	Exports oriented FDI			& 940		& 8\%	\\
	Technology intensive FDI 		& 1,555 	& 14\% 	\\
	Domestic market seeking FDI 	& 1,965 	& 17\% 	\\\\[-1.8ex]
	Total 							& 11,323 	& 100\% \\
	\bottomrule \bottomrule
\end{tabular}
\end{threeparttable}
\label{tab:freq}
\end{table}
%---------------------------

A set of categorical and continuous control variables was measured in 2015, one year prior to the firms receiving FDI. Table \ref{tab:cat} provides an overview of the categorical variables and the frequencies of each category in our sample. %Table \ref{tab:cat} provides an overview of the categorical variables and their frequencies in our sample. 
%The categorical variables are included as factor variables in the subsequent analysis. %leave out?
The port variable indicates whether a firm has access to a port within 500km. The legal ownership of a firm is captured in the ownership variable. The technology intensity of the industry the respective firm is operating in, is measured in four categories from low- to high-tech. The R\&D dummy indicates whether a firm has invested in Research and Development in 2015. \\

%-------------------------TABLE 2
\begin{table}[h!]
	\centering
	\caption{Summary Statistics of Categorical Covariates} 
	\begin{threeparttable}

\begin{tabular}{lcc} \toprule \toprule 
			& Abs. Freq.	& Rel. Freq. \\
	\midrule 							\\[-1.8ex]
\textbf{Port}\textsuperscript{a} & & 						\\
No				& 7,366		& 65.05		\\
Yes				& 3,957     & 34.95		\\[1.2ex]
\textbf{Ownership}	& &					\\
Listed company	& 909 		& 8.03 		\\
Subsidiary		& 2,630 	& 23.23 	\\
Independent 	&  4,593 	& 40.56 	\\
State owned		& 3,191     & 28.18		\\[1.2ex]
\textbf{Technology Intensity}	& &		\\
Low-tech		& 4,194		& 37.04 	\\
Medium low-tech	& 1,685		& 14.88 	\\
Medium high-tech & 3,539	& 31.25 	\\
High-tech  		& 1,905		& 16.82 	\\[1.2ex]
\textbf{R\&D}\textsuperscript{b}	& &						\\
No				&  9,951	& 87.88		\\
Yes				& 1,372		& 12.12 	\\[1.2ex]
	\hline \hline
\end{tabular}

\begin{tablenotes}[flushleft]
\footnotesize
\item \textsuperscript{a} Indicates whether a firm has access to a port within 500km.
\item \textsuperscript{b} Indicates whether a firm has invested in R\&D in 2015.
\end{tablenotes}

\end{threeparttable}
	\label{tab:cat}	
\end{table}
%--------------------------------

%-----------------------TABLE 3
\begin{table}[h!]
	\centering
	\caption{Summary Statistics of Continuous Covariates} 
	\begin{threeparttable}

\begin{tabular}{lccccc} 
\toprule \toprule
					& Mean 		& Median & Sd 		& Min 	& Max \\
\midrule \\[-1.8ex]
Wages 				& 1,967\textsuperscript{a} & 1,538 & 50,990\textsuperscript{a} & 0.00065 	& 5,519,000\textsuperscript{a} \\
TFP 				& 3.041 	& 3.032 & 2.047 	& -5.359 	& 11.36 	\\
Employment 			& 7,111 	& 81.39 & 117,155 	& 0.00197 	& 8,824\textsuperscript{a} \\
Debt 				& 1.762 	& 1.649 & 0.634 	& 0.819 	& 3.668 	\\
Export intensity 	& 0.159 	& 0.154 & 0.0798 	& 0.0103 	& 0.483 	\\\\[-1.8ex]
\bottomrule \bottomrule
\end{tabular}

\begin{tablenotes}[flushleft]
\footnotesize
\item \textit{Note:} All variables in levels.
\item \textsuperscript{a} In Thousands
\end{tablenotes}

\end{threeparttable}
	\label{tab:cont}
\end{table}
%------------------------------
\newpage

The summary statistics of the continuous variables, i.e. wages,  total-factor productivity (TFP), firm size\footnote{Since the original variable is only available in logarithmic form and lacks an indicator for the unit of measurement we assume it is measured in number of employees.}, debts and the firms' export intensity are displayed in Table \ref{tab:cont}. The variables wages, employment and, to a lesser extent, debts show large differences between their mean and median values, hinting at the existence of outliers in the sample. Taking the logarithm of the skewed variables reduces the influence of extreme values. However, including the log transformed employment variable yields worse covariate balance in all estimated models. We therefore include the untransformed employment variable in our models, despite noting at least one extreme value in this variable (see Figure \ref{fig:outliers}). We test the robustness of our models to the exclusion of observations with extreme values in the employment variable in section 4.

%ALT (saves 20 words): The variables wages and employment show large differences between their mean and median values. An easy way to reduce the influence of extreme values potentially causing this divergence is to take the logarithm of these variables. However, including the log transformed employment variable deteriorates the covariate balance in all estimated models. We therefore include the untransformed employment variable in the subsequent analysis, despite noting at least one extreme value in this variable (see Figure \ref{fig:outliers}). We test the robustness of our models to the exclusion of observations with extreme values in the employment variable in section 4.


%-------------------FIGURE 1
\begin{figure}[h!]\centering
	\caption{Outliers in Employment Variable}
	\includegraphics[width=\textwidth]{emp15_outliers}
  	\label{fig:outliers}
\end{figure} 
%---------------------------
\newpage

To further motivate the use of propensity scores in estimating the effect of FDI on a firm's TFP, we show the differences in means between  firms that received FDI and firms that did not in Table \ref{tab:meandiff1}. The t-tests show significant differences in all observable characteristics, suggesting that there is in fact selection into treatment. %OR: meaning that firms are in fact chosen to receive FDI based on specific characteristics.

%-------------------------------------TABLE 4

\begin{table}[h!]
	\centering
	\caption{Difference in Pre-Treatment Covariate Means}
	\makebox[\textwidth]{
	% Balancetable grapvar(FDI2016)

\begin{threeparttable}
\begin{tabular}{lccc}
\\[-1.8ex]\toprule \toprule \\[-1.8ex]
 & (1)  & (2)  & T-test  \\
 & Control  & Treatment  & Difference (1)-(2) \\ \midrule \\[-1.8ex] 
Technology intensity & 2.565 &  1.838 	& 0.728***		\\
				& (0.014) 	& (0.015) 	& 				\\
Access to port 	& 0.273 	& 0.467 	&  -0.194***	\\ 
				& (0.005) 	& (0.007)	& 				\\
Log wages		&  7.529 	& 7.031 	& 0.498*** 		\\ 
				& (0.046)	& (0.057) 	&  				\\
TFP 			& 3.185		& 2.821		& 0.364***		\\ 
				& (0.025) 	& (0.030)  	&				\\
Log employment 	& 3.766 	& 5.405 	& -1.639***		\\
				& (0.037) 	& (0.041)  	&				\\
Log debts 		& 0.511 	& 0.493 	& 0.019***		\\ 
				& (0.004) 	& (0.005)	&			   	\\
Export intensity &  0.131 	&  0.204 	& -0.073*** 	\\ 
				& (0.001) 	& (0.001) 	&				\\
R\&D dummy 		& 0.117 	& 0.128 	& -0.012*		\\ 
				& (0.004) 	& (0.005) 	&  				\\ \\[-1.8ex]
Observations 	& 6863 		& 4460 		&				\\
\bottomrule \bottomrule 
\end{tabular} 

\begin{tablenotes}[flushleft]
\footnotesize
\item \textit{Notes}: Columns (1) and (2) show the pre-treatment covariate means of the control and treatment group respectively. Standard errors are displayed in paratheses. The values displayed for t-tests are the differences in the means across the groups. ***, **, and * indicate significance at the 1, 5, and 10 percent critical level.
\end{tablenotes}

\end{threeparttable}

}
	\label{tab:meandiff1}
\end{table}
%------------------------------------------

\newpage
\section{Empirical Specification}

We have seen that FDI was not randomly assigned to firms. Thus, a simple comparison of treated and untreated firm outcomes will yield a biased treatment effect. %ALT:If FDI was not randomly assigned to firms, a simple comparison of treated and untreated firm outcomes will yield a biased treatment effect.
Instead, we use propensity score estimation to compare the outcomes of similar firms. For this purpose we estimate the likelihood of treatment for each firm, i.e. the propensity score. It is based on a set of observable characteristics that influence both the outcome and the likelihood of treatment. We assume that conditional on these confounders, the treatment is independent of the potential outcome, i.e. the Conditional Independence Assumption (CIA) is satisfied. 

%----------
For the main model we use the nearest-neighbour matching estimator, which compares the outcomes of treated observations to the closest control observation in terms of propensity scores. We estimate matching models  with one and five nearest neighbours with replacement. For the latter model we add a caliper cutoff at 0.05. We also fit inverse probability weighting models, which weigh observations by the inverse probability of being in their observed treatment group. Further, we estimate the treatment effect using the augmented inverse probability weighting model, which adds covariate adjustment to the weighing process. Thus, as long as either the propensity score or the covariate adjustment model is correctly specified, the results are unbiased \citep[p.~393]{imbens2015}. The point of using multiple estimators is  to ensure that the investigated effect is robust to the use of different estimation methods.  

We use the same specification of covariates for the matching, weighting and regression adjustment models, unless stated otherwise. We do not include the export variable as a matching covariate, assuming that exports do not increase productivity
%since only covariates that influence the likelihood of treatment and the outcome of interest should be included.
Although there is some debate about the direction of causality between exports and productivity, in his literature review \citet{wagner2007} argues that productivity increases exports, but not the other way around. The exclusion of the export variable significantly improves covariate balance. For the same reason we do not include the port variable.

Our propensity score model is thus a logit regression of binary treatment on ownership, technology intensity, a Research\&Development dummy, the logarithm of wages, TFP, employment and debts in 2015. Figure \ref{fig:graph} shows evidence of sufficient propensity score overlap for a matching analysis. The covariate balance of the different models is discussed in more detail below.
%-----------

%ALT (mainly different order):
\comment{
We use the same specification of covariates for all estimators, unless stated otherwise. We do not include the export variable as a matching covariate, assuming that exports do not increase firm productivity.
%since only covariates that influence the likelihood of treatment and the outcome of interest should be included.
Although there is some debate about the direction of causality between exports and productivity, in his literature review \citet{wagner2007} argues that productivity increases exports, but not the other way around. The exclusion of the export variable significantly improves covariate balance. We do not include the port variable for the same reason.

Our matching model is thus a logit regression of binary treatment %that is equal to one if the firm received FDI in 2016 
on ownership, technology intensity, a Research\&Development dummy, the logarithm of wages, TFP, employment and debts in 2015. Figure \ref{fig:graph} shows evidence of sufficient propensity score overlap for a matching analysis. %The covariate balance of the different models is discussed in more detail below.

For the main model we use the nearest-neighbour matching estimator, which compares the outcomes of treated observations to the closest control observations in terms of propensity scores. We estimate matching models  with one and five nearest neighbours with replacement. For the latter model we add a caliper cutoff at 0.05. We also fit inverse probability weighting models (IPW), which weigh observations by the inverse probability of being in their observed treatment group. Further, we estimate the treatment effect using the augmented inverse probability weighting model (AIPW), which adds covariate adjustment to the weighing. Thus, as long as either the propensity score or the covariate adjustment model is correctly specified, the results of the AIPW are unbiased \citep[p.~393]{imbens2015}. The point of using multiple estimators is to ensure that the investigated effect is robust to the use of different estimation methods.  }


%-------------------FIGURE 2
\begin{figure}[h!]
	\centering
	\caption{Propensity Score Overlap in Main Model}
	\includegraphics[width=\textwidth]{graph}
  	\label{fig:graph}
\end{figure} 
%---------------------------

\section{Results}

\subsection{Effect of FDI on TFP}
The main findings of this paper are displayed in Table \ref{tab:mainresults}. It reports the Average Treatment Effects of FDI on TFP. Across different estimators we find large and highly significant coefficients, indicating that receiving FDI increases TFP of companies on average. The reported coefficients differ only slightly in magnitude. 


%-------------------------------------TABLE 5
\begin{table}[h!]
 	\centering
   	\caption{ATE of FDI on TFP}
   	\label{tab:mainresults}
\begin{threeparttable}
	
 \begin{tabular}{l*{4}{c}}
	\hline
	\hline
 			& NN1 & NN5 & IPW & AIPW \\
 			& (1) & (2) & (3)  & (4) \\ \hline
 			&  &  &  &    \\
FDI2016 	& 0.130*** & 0.114*** & 0.122***  & 0.142***   \\
 			& (0.015) & (0.011) & (0.007) &   (0.003)  \\
 	&  &  &  &    \\
PO Means 	& & & -0.068*** &  -0.057*** \\
			&  &  & (0.010)  &  (0.009) \\
			&  &  &  &    \\
 Observations & 11,323 & 11,318 & 11,323 & 11,323 \\ 
 	\hline
 	\hline 
\end{tabular}

\begin{tablenotes}[flushleft]
      \footnotesize
\item \textit{Note}: This table reports the standardized coefficients of several matching estimators. All matching was done with replacement. Columns (1) and (2) show the coefficients of the one and five nearest neighbour propensity score matching respectively. For the NN5 matching, a caliper was set to .05. Columns (3) and (4) display the coefficients of the inverse probability and augmented inverse probability matching estimators respectively. The covariate adjustment model specification is the same as that of the propensity score model. Standard errors are displayed in parentheses. ***, **, and * indicate significance at the 1, 5, and 10 percent critical level.

\end{tablenotes}

\end{threeparttable}
\end{table}
%------------------------------------------

Column (1) shows the results of a one-to-one propensity score matching with replacement. Had all firms in our sample received FDI, the TFP would have increased by 13 percent of a standard deviation on average. Slightly lower results are obtained from a propensity score matching with five nearest neighbors and a caliper of 0.05 in column (2). The caliper cutoff excluded five observations. The estimate of the IPW in column (3) is also somewhat below that of column (1). The estimate of the doubly robust AIPW-estimator is slightly larger than that of the first model, but all estimates differ by no more than 3 percent of a standard deviation.

Checking the covariate balances of our models, the standardized differences and variance ratios are within a very good range for all models. We prefer the one-to-one propensity score matching as it gives us the best covariate balance of all the estimators. The maximum standardized difference among all covariates is four percent and the largest variance ratio is 1.7, with all others being close to one (see \ref{app:logfile}). 


\subsection{Robustness of Results}

\subsubsection*{Alternative Specifications} 
In order to test for the sensitivity of our main findings to alternative model specifications, we perform several robustness checks for the nearest-neighbour matching estimator with one neighbour. 
%ALT: We perform several robustness checks for the nearest-neighbour matching estimator with one neighbour to test for the sensitivity of our findings to alternative model specifications.
The results are reported in Table \ref{tab:robust}. The positive and significant effect of FDI on TFP persists through all specifications, confirming our main results that foreign investment increases the productivity of domestic firms. 

In column (1), we add interaction terms of the dummy variables with the continuous regressors in our propensity score model.%to our set of covariates. 
This is widely practiced to improve covariate balance \citep{Caliendo08}.
However, %the covariate balance of our model does not show 
in our case we do not find notable improvements but worse balances for some covariates. %with the inclusion of interaction terms. In fact, the covariate balance of the included interaction terms was not within an acceptable range, %suggesting that interactions do not increase the quality of matching.
\footnote{The same holds true when interacting only dummy variables, only continuous variables or all variables.} The estimated ATE of FDI on productivity slightly increases by 0.022 standard deviations compared to the effect reported in Table \ref{tab:mainresults}. 

As we decided not to log transform the employment variable, our results could further be biased by its outliers (see Figure \ref{fig:outliers}). While most of the firms' employee numbers are concentrated around the mean of 7,111, we are concerned about two observations with extreme values, with eight and four million employees respectively.
%one firm with over eight million employees, and another one that, according to the data, employed more than four million people in 2015. 

To check whether these outliers influence our main findings, we restrict the sample to firms with less than four million employees. The results reported in column (2) show no significant change in the treatment effect when excluding the two extreme observations. 

We have also assumed that the presence of a port within 500 km of the firm does not influence productivity. Although one could argue that having access to a port might increase productivity e.g. by facilitating market access, we find no evidence of this.  Column (3) reports only a small change in the estimate of 0.5 percent of a standard deviation when including the port dummy in our set of covariates. 

Column (4) reports the Average Treatment Effect on the Treated (ATT) of the propensity score matching with one neighbor and replacement. While the ATE measures the average effect of FDI for the hypothetical case that all firms received FDI, the ATT estimates the effect only on those firms that actually received treatment. Because it is assumed that selection into treatment is non-random, we might find different effects of treatment on the treated. 
It could, for example, be higher if those firms receiving treatment are also the ones benefiting more from it in terms of productivity. %leave out?
 However, our estimate in column (4) reports an ATT that is very similar to the average treatment effect. This suggests that although there was selection into treatment, the effect size would be very similar in the absence of such selection. %ALT: This suggests that although there was selection into treatment, our propensity score model yields similar results as under randomization, where ATE and ATT are equal.
 
%-----------------------TABLE 6
\begin{table}[h]
  \centering
   \caption{Robustness of Results}
   \label{tab:robust}
\begin{threeparttable}
 
\begin{tabular}{lcccc} 
	\hline 
	\hline
 		& & & & Effect \\
 		& Including & Excluding & Including & on the\\
 		& Interactions & Outliers 
 		& Port & Treated \\
 		& (1) & (2) & (3) & (4) \\ 
 	\hline
 		&  &  &  &  \\
ATE 	& 0.152*** & 0.127*** & 0.125*** &  \\
 		& (0.016) & (0.015) & (0.019) &  \\
 		&  &  &  &  \\
ATT 	&  &  &  & 0.127*** \\
 		&  &  &  & (0.017) \\
 		&  &  &  &  \\
Observations & 11,323 & 11,321 & 11,323 & 11,323 \\ 
	\hline
	\hline
\end{tabular}

\begin{tablenotes}[flushleft]
     \footnotesize  
    
\item \textit{Note}: All specifications are variations of our main model using the Propensity Score Matching method with one nearest neighbour and replacement. Covariates in the main model included: Ownership, Technology Intensity, Research\&Development, logarithm of Wages, Total Factor Productivity, Employment and Debts. In column (1), the main model is augmented by interactions of the dummy variables (Ownership, Technology Intensity, Research \& Development) with continuous variables (Logarithm of wages, Total Factor Productivity, Employment and Debts).  The specification in column (2) excludes two observations with values of Employment 2015 above four million. In column (3) we include a dummy variable indicating whether a port lies within 500km of the firm as an additional covariate. Column (4) reports the average treatment effect on the treated. Standard errors are displayed in parentheses. ***, **, and * indicate significance at the 1, 5, and 10 percent critical level. 

\end{tablenotes}

\end{threeparttable}
\end{table}
%--------------------------------------

\subsubsection*{Effects by Technology Intensity}

FDI flows vary strongly between different sectors (see, for example, \citet{Smarzynska2004, Keller2009, Haskel2007}). In our sample, firms are divided into four industry groups, ranging from low-tech to high-tech industries. While foreign investors target only 13 percent of firms in high-tech industries, more than half of the firms in low-tech industries have received FDI in 2016.\footnote{See Appendix \ref{app:tech}.} There is in fact some empirical evidence for heterogeneity of the effect of FDI on firm productivity depending on a firm's technology intensity. %ALT: Empirical evidence suggests that the effect of FDI on firm productivity is heterogeneous, depending on a firm's technology intensity.
For instance, \citet{Keller2009} find a strong effect of FDI on the productivity of domestically owned firms in the high-tech sector but only a very small, if any, effect on low-tech industries. 
To test for this possibility, we estimate the ATE of FDI on productivity separately for each industry and report the results in Table \ref{tab:TECH}. Standard errors have increased slightly, but the results are still highly significant. 

%-----------------------TABLE 7
\begin{table}[h!]
  \centering
   \caption{ATE by Technology Intensity of Industry}
   \label{tab:TECH}
\begin{threeparttable}
 
\begin{tabular}{lcccc}
 \hline
 \hline
 & & Medium & Medium &  \\ 
 & Low-Tech & Low-Tech & High-Tech & High-Tech \\ 
 & Industry & Industry & Industry & Industry \\ 
 & (1) & (2) & (3) & (4) \\
 \hline
 &  &  &  &  \\
FDI2016 & 0.160*** & 0.086*** & 0.172*** & 0.180*** \\
	      & (0.020) & (0.028) & (0.019) & (0.054) \\
	      &  &  &  &  \\
 Observations & 4,194 & 1,685 & 3,539 & 1,905 \\ 
	\hline
	\hline
\end{tabular}	

\begin{tablenotes}[flushleft]
     \footnotesize       
\item \textit{Note}: The table reports the standardized ATE coefficients for subsamples of firms with different levels of technology intensitiy. Standard errors are displayed in paratheses. ***, **, and * indicate significance at the 1, 5, and 10 percent critical level. 

\end{tablenotes}


\end{threeparttable}
\end{table}
%--------------------------------------

The impact of FDI does indeed vary across industries. Our estimates support the finding of \citet{Keller2009} that firms in high-tech industries benefit the most, as FDI increases productivity of these firms by 18 percent of a standard deviation, five percentage points more than our results for the full sample would suggest. Somewhat surprising is that the estimates for the low-tech industry are also higher than in our main specification. The medium low-tech industry, instead, benefits much less than the other industries. It experiences an increase in TFP of only 8.6 percent of a standard deviation when receiving FDI. \\

The weighted average of these estimates yields an ATE of FDI on TFP of 0.158 standard deviations.\footnote{Weights are allocated according to relative subsample size.} This effect slightly differs from our main result due to the fact that matching is now performed within industry only. Although matched neighbours might be more 'distant' regarding other covariate values, we can ensure that each treated firm is allocated to a control observation with the same technology intensity. Despite the smaller sample sizes, the covariate balances remain good overall. 


\subsection{Analysis by Type of FDI}

%To further test the robustness of our results, %leave out?
We continue our analysis by looking at potential heterogeneity of the treatment effect across types of FDI. We test the possibility that one specific type of investment single handedly drives our previous results. It is possible that, for example, only exports-oriented FDI increases factor productivity while the other two types have little or no impact. This would violate the Stable Unit Treatment Value Assumption (SUTVA), necessary for causal effect stability. 

We estimate an augmented IPW model with multi-valued treatment effects. The propsensity score and regression adjustment model specifications are the same as that of our main mdoel. The model yields good covariate balance.
We further estimate an IPW model to check if it returns similar estimates without regression adjustment. The covariate balance in this model is practically the same. Finally, we specify a set of AIPW models, each comparing only one type of treatment to non-treated observations. This allows for the IIA assumption to be relaxed which is required for mulitnomial logit models. The separate models have worse covariate balance than the first two but are still acceptable. The overlap assumption is satisfied for all treatment levels as can be seen in Figure \ref{fig:over_typ}. 

In Table \ref{tab:bytype} the results from the type-wise analysis are shown. In the AIPW multinomial specification, the ATE of different types of FDI are within half a percent of each other. This suggests that all types of FDI increase factor productivity by the same margin. The estimated effect size is close to the one estimated for FDI in Table \ref{tab:mainresults}. In the IPW specification the differences are slightly larger but still within 5 percent of a standard deviation of each other.  The separate logit models also yield essentially the same effect sizes as the multinomial specification. Since the AIPW estimator is doubly robust, assuming correctly specified covariate adjustment models, the results in columns (1) and (3) provide us with strong evidence of homogenous effects of different FDI Types on TFP.


%-------------------FIGURE 2
\begin{figure}[h]
	
	\caption{Propensity Score by Treatment Level}
\hspace*{-2cm}  	
	\includegraphics[height=12cm]{overlap_type.png}\\ 
	\label{fig:over_typ}
 
\end{figure}
%---------------------------



%-----------------------------TABLE 8
\begin{table}[htbp]
	\centering
	\caption{ATE by Type of FDI}
	\label{tab:bytype}
\begin{threeparttable}

\begin{tabular}{lccccc} 
		\hline
		\hline
 	& (1) & (2) & (3) & (4) & (5) \\
	& AIPW & IPW  & AIPW  & AIPW & AIPW \\ 
	& Mlogit & Mlogit &Logit &Logit &Logit\\
		\hline
 			&  &  &  &  &   \\
Exports-oriented FDI 	& 0.144*** &   0.157*** & 0.140*** &  &  \\
 						& (0.006) &   (0.032) & (0.007) &  &\\ \\[-1.8ex]
Technology intensive FDI & 0.139***   & 0.112*** &  & 0.139*** &   \\
 						 & (0.005)  & (0.018) &  &  (0.005)&  \\ \\[-1.8ex]
Domestic market seeking FDI & 0.143*** &   0.134*** &  &  &0.143*** \\
 							& (0.004)   & (0.011) &  &  & (0.004)  \\ \\[-1.8ex]
PO Means 		&   -0.057*** &   -0.068*** &-0.012  &-0.025**  & -0.017    \\
 				&   (0.009) &   (0.010) &  (0.011)&(0.011)  & (0.011) \\ \\[-0.2em]
Observations 	& 11,323  & 11,323 &  7,803  & 8,418 & 8,828  \\ 
		\hline
		\hline
\end{tabular}

\begin{tablenotes} [flushleft]
\footnotesize
\item \textit{Note:} Columns (1) and (2) report the coefficients of the multinominal augmented inverse probability and multinominal inverse probability matching estimators respectively. Columns (3)-(5) display the results of the augmented inverse probability matching estimator for subsamples of firms having received different types of FDI. Standard errors are displayed in paratheses. ***, **, and * indicate significance at the 1, 5, and 10 percent critical level. 
\end{tablenotes}

\end{threeparttable}
\end{table}
%------------------------------------------------


\section{Conclusion}

Using Propensity Score-based estimators, we find a positive, economically and statistically significant effect of FDI on firm productivity. This effect is robust across various estimators as well as to different model specifications. However, we find evidence of heterogeneity across technology levels. 
In contrast to previous findings, this effect is not linearly increasing with technology intensity. %leave out?
The treatment effect is essentially the same for all types of FDI. 
%Regardless of the type of FDI a firm receives, the effect remains positive and essentially the same size.

While our findings are broadly in line with the empirical literature, our ability to contextualize our results is limited by the lack of information about our data. For example, our dataset does not provide a detailed industry classification of firms, or their geographical location. These variables might influence both treatment and outcome and would thus have been included in our analysis. Moreover, we can only report estimates of the initial impact of FDI on TFP in the year after treatment. Thus we cannot make any claims about the persistence of the effect. %As mentioned in the literature review, the effects of FDI seem to change over time. 
A DiD-Matching combination would have allowed us to control for the %possible impact of 
unobservable firm characteristics, however, this would exceed the scope of our analysis. Finally, due to the limited information on a firm's sector and location, we are unable to account for spillover effects on nearby firms or on firms within the same industries. This may lead to an underestimation of the ATE. 


\newpage

%-------------------------------------------------------------
% References
%-------------------------------------------------------------
\addcontentsline{toc}{section}{References}	%Adds references to table of contents
\bibliography{bibCG} 
\newpage


%-------------------------------------------------------------
% Appendix
%-------------------------------------------------------------
\pagenumbering{roman}
\sectionnumbering{Roman}
\setcounter{page}{3} %May have to adjust this if we leave out list of tables or add something else

\appendix
\section{Appendix}

%---------------------------FDI BY TECH (ROBUSTNESS)
\subsection{Treatment by Technology Intensity}
\label{app:tech}
\begin{table}[htbp]
	\centering
\begin{threeparttable}

\begin{tabular}{lcccccc} 
\hline
\hline
 & \multicolumn{6}{c}{FDI in 2016} \\
Technology intensity & \multicolumn{3}{c}{No} & \multicolumn{3}{c}{Yes} \\
of industry &No.&Col \% &Cum \% &No.&Col \% &Cum \% \\
\hline
	&  &  &  &  &  &   \\
Low-tech &1869&44.6&27.2&2325&55.4&52.1 \\
Medium low-tech &904&53.6&40.4&781&46.4&69.6 \\
Medium high-tech &2432&68.7&75.8&1107&31.3&94.5 \\
High-tech &1658&87.0&100.0&247&13.0&100.0 \\
\textbf{Total}&\textbf{6863}&\textbf{60.6}&&\textbf{4460}&\textbf{39.4}& \\
\hline
\hline
\end{tabular}

\end{threeparttable}
\end{table}
%-----------------------------------

\subsection{Stata Output}
\label{app:logfile}

\end{document}