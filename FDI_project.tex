% -------------------------------------------------------------
%		Preamble
%--------------------------------------------------------------
% !TeX spellcheck = uk_UK
%Document Settings
\documentclass[a4paper,11pt]{scrartcl}
\usepackage[T1]{fontenc}
\usepackage{setspace}
\onehalfspacing
\DeclareOldFontCommand{\bf}{\normalfont\bfseries}{\mathbf} % Includes an old command that is part of some package in here (just to make sure everything runs)

% Page numbers
\newcommand{\sectionnumbering}[1]{%
  \setcounter{section}{0}%
   \renewcommand{\thesection}{\csname #1\endcsname{section}}}

% Language settings
\usepackage[utf8]{inputenc}
\usepackage[english]{babel}

% Maths settings
\usepackage{amsmath}
\newcommand\numberthis{\addtocounter{equation}{1}\tag{\theequation}}
\usepackage{amssymb}
\usepackage{siunitx}

% Graphics packages
\usepackage{graphicx}
\graphicspath{{Graphics/}}
\usepackage{adjustbox}

% Table packages
\usepackage{pdflscape} %To create landscape environments
\usepackage{booktabs,caption}
\usepackage{threeparttable} %For nice tables
\usepackage{csvsimple} %To import excel tables 
\usepackage{import}

% Bilbiography settings
\usepackage{natbib}
\bibliographystyle{apalike}

%Commenting over multiple rows
\newcommand{\comment}[1]{}

% -------------------------------------------------------------
%		Title Page
%--------------------------------------------------------------
\begin{document}

	\begin{titlepage}
		\newcommand{\HRule}{\rule{\linewidth}{0.5mm}}
		
%% Logo
	\vfill\vfill
	\includegraphics[height=1.5cm]{UoN_Logo}\\[1cm] 


	\center			
%% Heading
	\textsc{\LARGE University of Nottingham}\\[1.5cm] 
	\textsc{\Large Applied Microeconometrics}\\[0.5cm] 	
	\textsc{\large Group Project A}\\[0.5cm] 
	
%% Title
	\HRule\\[0.4cm]
	{\huge\bfseries Insert Title}\\[0.4cm] 
	\HRule\\[0.4cm]
	
%% Date
	{\large\ Spring Term 2020} 	
	\vfill\vfill\vfill 		
	
%% Author(s) and Supervisor
\begin{flushleft}
			\large
			\textit{Supervisor}\\
			Professor Sourafel \textsc{Girma} 
			\vfill\vfill 
			\textit{Authors}\\
			Emilie \textsc{Bechtold} (20214031)\\
			Nelly  \textsc{Lehn} (20214338)\\
			Yonesse \textsc{Paris} (20115536)\\
			Georg  \textsc{Schneider} (20214032)\\
			Thea  \textsc{Zoellner} (20216019)
		\end{flushleft}
	\vfill 
	
\end{titlepage}


% -------------------------------------------------------------
%		Contents
%--------------------------------------------------------------
\pagenumbering{roman}
\sectionnumbering{Roman}
\tableofcontents

\newpage

\listoftables
\newpage

%-------------------------------------------------------------
% Main Body
%-------------------------------------------------------------
\pagenumbering{arabic}
\sectionnumbering{arabic}

\section{Introduction}



\section{Theoretical Background/Literature Review}

\subsection{FDI}

\subsection{PSM}
Since (I guess) we will be focussing on ATE rather than ATT, we need to satisfy the following two assumptions: 

\begin{enumerate}
\item Assumption: \textbf{Unconfoundedness (CIA)} \\
"\textit{[G]iven a set of observable covariates X which are not affected by treatment, potential outcomes are independent of treatment assignment}"   \citep[p.~35]{CaliendoHujerThomsen2008}	 

\item Assumption: \textbf{Overlap} \\
"\textit{persons with the same X values have a positive probability of being both participants and nonparticipants}" \citet [p.~35]{Caliendo08}

\end{enumerate}
--> if Assumption 1 holds, all biases due to observable components can be removed by conditioning on the propensity score (Imbens, 2004).

\subsubsection*{Binary Treatment}
Difference between logit and probit lies in the link function. Logit assumes a log-distribution of residuals, probit assumes a normal distribution. Heteroskedastic probit models can account for non-constant error variances --> Check for heteroskedasticity?

\subsubsection*{Multiple Treatments}
The multinomial probit model is the preferable option compared to logit. Alternatively, just run several binary ones (more complicated but also more robust to errors).

\subsubsection*{Variable selection}
\begin{itemize}
\item outcome variable must be independent of treatment conditional on the pscore (CIA)
\item Only variables that influence simultaneously the participation decision and the outcome variable should be included (based on theory and empirical findings)
\item variables should either be fixed over time or measured before participation (include only variables unaffeted by participation)
\item choice of variables should be based on economic theory and previous empirical findings
\end{itemize}

\subsubsection*{Tests for variable selection}
Strategies for the selection of variables to be used in estimating the propensity score: ...


\section{Data and Descriptive Analysis}
Our analysis is based on observational firm-level data. The dataset comprises 11,323 firms, of which 4,460 received FDI in 2016. The FDI can be divided into three subcategories. Table \ref{tab:freq} shows the frequencies of each type of FDI in our sample. Among the recipients of FDI, most firms (1,965) received domestic market seeking FDI. 1,555 firms received technology intensive FDI and the remaining 640 firms received exports oriented FDI. The TFP outcome was measured in 2017. We standardize \footnote{Using Z-score standardization.} the outcome variable to make its interpretation more intuitive. % We standardize the outcome to a mean of 0 and a standard deviation of 1.
%first time writing FDI in document: 'foreign direct investment (FDI)'

%--------------------TABLE 1
\begin{table}[h]
	\centering
	\caption{Frequency of FDI Types} 
	\begin{threeparttable}
\begin{tabular}{lcc} 
	\toprule \toprule 
	FDI type&Abs. Freq.&Rel. Freq. \\[1ex]
	\midrule
	No FDI							& 6,863		& 61\% \\
	Exports oriented FDI			& 940		& 8\%	\\
	Technology intensive FDI 		& 1,555 	& 14\% 	\\
	Domestic market seeking FDI 	& 1,965 	& 17\% 	\\\\[-1.8ex]
	Total 							& 11,323 	& 100\% \\
	\bottomrule \bottomrule
\end{tabular}
\end{threeparttable}
\label{tab:freq}
\end{table}
%---------------------------

A set of categorical and continuous control variables was measured in 2015 (one year prior to the firms receiving FDI). Table \ref{tab:cat} provides an overview of the categorical variables and the frequencies of each category in our sample. The categorical variables are included as factor variables in the subsequent analysis. The port variable indicates whether a firm has access to a port within 500km. The legal ownership of a firm is captured in the ownership variable, where state owned firms represent the base group. The technology intensity of the industry the respective firm is operating in, is measured in four categories from low- to high-tech, low-tech being the base group. The R\&D dummy indicates whether a firm has invested in Research and Design in 2015. \\

%-------------------------TABLE 2
\begin{table}[h]
	\centering
	\caption{Summary Statistics of Categorical Covariates} 
	\begin{threeparttable}

\begin{tabular}{lcc} \toprule \toprule 
			& Abs. Freq.	& Rel. Freq. \\
	\midrule 							\\[-1.8ex]
\textbf{Port}\textsuperscript{a} & & 						\\
No				& 7,366		& 65.05		\\
Yes				& 3,957     & 34.95		\\[1.2ex]
\textbf{Ownership}	& &					\\
Listed company	& 909 		& 8.03 		\\
Subsidiary		& 2,630 	& 23.23 	\\
Independent 	&  4,593 	& 40.56 	\\
State owned		& 3,191     & 28.18		\\[1.2ex]
\textbf{Technology Intensity}	& &		\\
Low-tech		& 4,194		& 37.04 	\\
Medium low-tech	& 1,685		& 14.88 	\\
Medium high-tech & 3,539	& 31.25 	\\
High-tech  		& 1,905		& 16.82 	\\[1.2ex]
\textbf{R\&D}\textsuperscript{b}	& &						\\
No				&  9,951	& 87.88		\\
Yes				& 1,372		& 12.12 	\\[1.2ex]
	\hline \hline
\end{tabular}

\begin{tablenotes}[flushleft]
\footnotesize
\item \textsuperscript{a} Indicates whether a firm has access to a port within 500km.
\item \textsuperscript{b} Indicates whether a firm has invested in R\&D in 2015.
\end{tablenotes}

\end{threeparttable}
	\label{tab:cat}	
\end{table}
%--------------------------------

The summary statistics of the continuous variables wages, Total Factor Productivity (TFP), firm size (measured in number of employees), debts and the firms' export intensity are displayed in Table \ref{tab:cont}. The variables wages, employment and, to a lesser extent, debts show large differences between their mean and median values, hinting at the existence of outliers in the sample. Log transforming the skewed variables is an easy way of reducing the influence of extreme values. 

%-----------------------TABLE 3
\begin{table}[h]
	\centering
	\caption{Summary Statistics of Continuous Covariates} 
	\begin{threeparttable}

\begin{tabular}{lccccc} 
\toprule \toprule
					& Mean 		& Median & Sd 		& Min 	& Max \\
\midrule \\[-1.8ex]
Wages 				& 1,967\textsuperscript{a} & 1,538 & 50,990\textsuperscript{a} & 0.00065 	& 5,519,000\textsuperscript{a} \\
TFP 				& 3.041 	& 3.032 & 2.047 	& -5.359 	& 11.36 	\\
Employment 			& 7,111 	& 81.39 & 117,155 	& 0.00197 	& 8,824\textsuperscript{a} \\
Debt 				& 1.762 	& 1.649 & 0.634 	& 0.819 	& 3.668 	\\
Export intensity 	& 0.159 	& 0.154 & 0.0798 	& 0.0103 	& 0.483 	\\\\[-1.8ex]
\bottomrule \bottomrule
\end{tabular}

\begin{tablenotes}[flushleft]
\footnotesize
\item \textit{Note:} All variables in levels.
\item \textsuperscript{a} In Thousands
\end{tablenotes}

\end{threeparttable}
	\label{tab:cont}
\end{table}
%------------------------------

However, during placebo estimation with a different outcome it became clear that not taking the logarithm of employment yields better covariate balance in all models. Therefore we use the untransformed variable. Noting an extreme value in the variable (see Figure \ref{fig:outliers}) we added a robustness check that excluded this observation, yielding essentially the same results. 
 

%-------------------FIGURE 1
\begin{figure}[h]\centering
	\caption{Outliers in Employment Variable}
	\includegraphics[width=\textwidth]{emp15_outliers}
  	\label{fig:outliers}
\end{figure} 
%---------------------------
%NOTE: change font to same as Latex uses

To further motivate the use of matching estimators in estimating the effect of FDI on a firm's TFP, the differences in means between the firms that received FDI and the firms that did not are displayed in Table \ref{tab:meandiff1}. The t-tests show significant differences in all observable characteristics, meaning that there is in fact selection into treatment.

%-------------------------------------TABLE 4
\begin{table}[hbtp!]
	\centering
	\caption{Difference in Pre-Treatment Covariate Means}
	% Balancetable grapvar(FDI2016)

\begin{threeparttable}
\begin{tabular}{lccc}
\\[-1.8ex]\toprule \toprule \\[-1.8ex]
 & (1)  & (2)  & T-test  \\
 & Control  & Treatment  & Difference (1)-(2) \\ \midrule \\[-1.8ex] 
Technology intensity & 2.565 &  1.838 	& 0.728***		\\
				& (0.014) 	& (0.015) 	& 				\\
Access to port 	& 0.273 	& 0.467 	&  -0.194***	\\ 
				& (0.005) 	& (0.007)	& 				\\
Log wages		&  7.529 	& 7.031 	& 0.498*** 		\\ 
				& (0.046)	& (0.057) 	&  				\\
TFP 			& 3.185		& 2.821		& 0.364***		\\ 
				& (0.025) 	& (0.030)  	&				\\
Log employment 	& 3.766 	& 5.405 	& -1.639***		\\
				& (0.037) 	& (0.041)  	&				\\
Log debts 		& 0.511 	& 0.493 	& 0.019***		\\ 
				& (0.004) 	& (0.005)	&			   	\\
Export intensity &  0.131 	&  0.204 	& -0.073*** 	\\ 
				& (0.001) 	& (0.001) 	&				\\
R\&D dummy 		& 0.117 	& 0.128 	& -0.012*		\\ 
				& (0.004) 	& (0.005) 	&  				\\ \\[-1.8ex]
Observations 	& 6863 		& 4460 		&				\\
\bottomrule \bottomrule 
\end{tabular} 

\begin{tablenotes}[flushleft]
\footnotesize
\item \textit{Notes}: Columns (1) and (2) show the pre-treatment covariate means of the control and treatment group respectively. Standard errors are displayed in paratheses. The values displayed for t-tests are the differences in the means across the groups. ***, **, and * indicate significance at the 1, 5, and 10 percent critical level.
\end{tablenotes}

\end{threeparttable}


	\label{tab:meandiff1}
\end{table}
%------------------------------------------


\section{Empirical Specification}

In order to evaluate the causal effect of FDI on Total Factor Productivity (TFP) we estimate our main model described in equation (1) using nearest-neighbour matching with one neighbour and with replacement. Second, we also estimate a nearest neighbour model with 5 neighbours with replacement and a caliper of 0.05. Further, we use  inverse probability weighting (IPW) and augmented-inverse probability weighting (AIPW) estimators. Since we have a large sample, the different estimators should all yield essentially the same results.
%Note: Second but no 'first'
For the matching,  weighting  and regression adjustment models we use the same specification including the before-mentioned categorical and continuous variables. We unlog employment to improve covariate balance. We do not include the export variable as a matching covariate, since together with technology it predicts treatment too well and nearly eliminates overlap. 
Propensity score matching requires that all variables which affect both outcome and likelihood of treatment are included as covariates. As Girma et al. (2004) show, productivity increases exports, not the other way around. %ALT: but the level of exports has no effect on productivity. Thus, we can exclude it.
We do not include the port variable for the same theoretical reason. 

Our matching model is thus a logit regression of TFP in 2017 on Ownership, Technology Intensity, Research\&Development, logarithm of Wages, TFP in 2015, Employment and Debts. This yields a very good overlap and covariate balance (the latter discussed in more detail below). Figure \ref{fig:overlap} shows proof of sufficient overlap for a matching analysis. 

%-------------------FIGURE 2
\begin{figure}[h]\centering
	\caption{Overlap in Main Model}
	\includegraphics[width=\textwidth]{overlap}
  	\label{fig:overlap}
\end{figure} 
%---------------------------
% double headline should be removed

\subsection{Effect of FDI on TFP}

%-------------------------------------TABLE 5

\begin{table}[h]
 	\centering
   	\caption{ATE of FDI on TFP}
   	\label{tab:mainresults}
\begin{threeparttable}

\begin{tabular}{lcccc} 
	\hline
	\hline
 			& NN1 & NN5\textsuperscript{a} & IPW & AIPW \\
 			& (1) & (2) & (3)  & (4) \\ \hline
 			&  &  &  &    \\
FDI2016 	& 0.130*** & 0.114*** & 0.122***  & 0.142***   \\
 			& (0.015) & (0.011) & (0.007) &   (0.003)  \\
 	&  &  &  &    \\
PO Means 	& & & -0.068*** &  -0.057*** \\
			&  &  & (0.010)  &  (0.009) \\
			&  &  &  &    \\
 Observations & 11,323 & 11,318 & 11,323 & 11,323 \\ 
 	\hline
 	\hline
	\multicolumn{5}{c}{\footnotesize{Standard errors in parentheses. *** p$<$0.01, ** p$<$0.05, * p$<$0.1. }}
\end{tabular}

\begin{tablenotes}[flushleft]
      \footnotesize
\item \textit{Note}: Covariates include, unless otherwise specified: Ownership, Technology Intensity, Research\&Development, logarithm of wages, Total Factor Productivity, Employment and Debts. 
\item All reported coefficients are normalized for better interpretation. %all coefficients or only dependent variable?
\item\textsuperscript{a} We use the Propensity Score Matching method with five nearest neighbours and replacement as well as a 0.05 caliper. 
\end{tablenotes}

\end{threeparttable}
\end{table}
%------------------------------------------

The main findings of this paper are displayed in Table \ref{tab:mainresults}. It reports the Average Treatment Effects (ATE) for FDI on TFP. Across estimators, we find large and highly significant coefficients showing that receiving FDI increases TFP of companies on average. The reported coefficients differ only slightly in magnitude, suggesting that the matching model is correctly specified. 
Column (1) shows the results of a one-to-one propensity score matching with replacement. Receiving FDI  increases factor productivity by 13\% of a standard deviation for the average company. Slightly lower results are obtained from a propensity score matching with five nearest neighbors and a caliper of 0.05 as well as for the inverse IPW. The doubly robust AIPW-estimator is slightly larger than the main model, but the estimators are within three percent of each other.

Checking the covariate balances of our models, we prefer the one-to-one propensity score matching as it gives us the best covariate balance of all the estimators \textit{see Appendix}. %In the next subsection, we probe the sensitivity of your findings in more detail for one-to-one propensity score matching. 


\subsection{Robustness of Results}

In order to test for the sensitivity of our main findings to alternative model specifications, we perform several robustness checks for the nearest-neighbour matching estimator with one neighbour. The results are reported in Table \ref{tab:robust}. The positive and significant effect of FDI on TFP persists through all specifications, confirming our main results that foreign equity participation increases the productivity of domestic firms. 

In column (1), we add interaction terms of the dummy variables with the continuous regressors to our set of covariates. This is widely practiced to improve covariate balance \citep{Caliendo08}.
However, the covariate balance of our model does not improve with the inclusion of interaction terms, suggesting that interactions do not increase the quality of matching.\footnote{The same holds true when interacting only dummy variables, only continuous variables or all variables.} The estimated ATE of FDI on productivity slightly increases by 0.022 standard deviations compared to the effect reported in Table \ref{tab:mainresults}. 

Our results could further be biased by outliers of the employment variable (see Figure \ref{fig:outliers}). While most of the firms' employee numbers are concentrated around the mean of 7,111, we are concerned about two observations with extreme values: one firm with over eight million employees, and another one that apparently employed more than four million people in 2015.\footnote{Given the limited information our dataset contains, we cannot be sure in which unit employment is measured. We therefore suppose that the common definition of employment being the number of hired employees holds. Regardless of the measurement unit of employment, the outliers might be problematic.}  To check whether these outliers influence our main findings, we restrict the sample to firms with less than four million employees. The results reported in column (2) show no significant change in the treatment effect when excluding the two extreme observations. %Note% should do F-test for equality of coefficients?%
Besides, we have assumed that the presence of a port within 500 km of the firm does not influence productivity. Column (3) reports a small change of 0.005 standard deviations when including the dummy variable port in our set of covariates. 
Column (4) reports the average treatment effect on the treated (ATT) of the propensity score matching with one neighbor and replacement. While the ATE measures the average effect of FDI for the hypothetical case that all firms have received FDI, the ATT only considers those firms that have actually been treated. Because it is assumed that selection into treatment is non-random, we might find stronger effects of treatment on the treated. This would be the case if those firms receiving treatment are also the ones benefiting more from it. However, our estimate in column (4) reports an ATT that is very similar to the average treatment effect. This suggests that although there was selection into treatment, the effect size would be very similar in the absence of such selection. 

%-----------------------TABLE 6
\begin{table}[h]
  \centering
   \caption{Robustness of Results}
   \label{tab:robust}
\begin{threeparttable}
 
\begin{tabular}{lcccc} 
	\hline 
	\hline
 		& & & & Effect \\
 		& Including & Excluding & Including & on the\\
 		& Interactions\textsuperscript{a} & Outliers\textsuperscript{b} 
 		& Port\textsuperscript{c} & Treated \\
 		& (1) & (2) & (3) & (4) \\ 
 	\hline
 		&  &  &  &  \\
ATE 	& 0.152*** & 0.127*** & 0.125*** &  \\
 		& (0.016) & (0.015) & (0.019) &  \\
 		&  &  &  &  \\
ATT 	&  &  &  & 0.127*** \\
 		&  &  &  & (0.017) \\
 		&  &  &  &  \\
Observations & 11,323 & 11,321 & 11,323 & 11,323 \\ 
	\hline
	\hline
	\multicolumn{5}{c}{\footnotesize{Standard errors in parentheses. *** p$<$0.01, ** p$<$0.05, * p$<$0.1. }}
\end{tabular}

\begin{tablenotes}[flushleft]
     \footnotesize  
     
\item \textit{Note}: All specifications use the Propensity Score Matching method with one nearest neighbor and replacement. Covariates include, unless otherwise specified: Ownership, Technology Intensity, Research\&Development, logarithm of Wages, Total Factor Productivity, Employment and Debts. 

\item\textsuperscript{a} Interacting dummy variables (Ownership, Technology Intensity, Research \& Development) with continuous variables (Logarithm of wages, Total Factor Productivity, Employment and Debts).  

\item\textsuperscript{b} Two observations with extreme values of Employment 2015 above four million are excluded.
 
\item\textsuperscript{c} Adding a dummy variable indicating whether a port lies within 500km of the firm to the set of covariates.
\end{tablenotes}

\end{threeparttable}
\end{table}
%--------------------------------------

\subsection{Treatment Effects by Technology Intensity}

FDI flows vary strongly between different sectors (see, for example, \citet{Smarzynska2004, Keller2009, Haskel2007}). In our sample, firms are divided into four industry groups, ranging from low-tech to high-tech industries. While foreign investors targeted only 13 percent of firms in high-tech industries, more than half of the observations in low-tech industries have received FDI in 2016.\footnote{See Appendix \ref{app:tech}.} 
This selection can have several reasons that we do not discuss in further detail. Instead, we focus on the change in the ATE when analyzing industries separately. For instance, \citet{Keller2009} find a strong effect of FDI on the productivity of domestically owned firms in the high-tech sector but only a very small, if any, effect on low-tech industries. In Table \ref{tab:TECH}, we report the estimates for the ATE of FDI on productivity separately for each industry. Standard errors have increased slightly, but the results are still highly significant. The impact of FDI does indeed vary across industries.  Our estimates support the finding of \citet{Keller2009} that firms in high-tech industries benefit the most, as FDI increases productivity of these firms by 18 percent of a standard deviation, five percentage points more than our results for the full sample suggest. Somewhat surprising is that the estimates for the low-tech industry are also higher than in our main specification. The medium low-tech industry instead does not benefit as much as the other industries do. It experiences an increase in TFP of only 8.6 percent when receiving FDI. 
%An F-test rejects the null hypothesis of equality of the ATE between industries. (Probably, if I figured out how to do it...)
\\
The weighted average of these estimates yields an ATE of FDI on TFP of 0.158 standard deviations.\footnote{Weights are allocated according to relative sample size.} This effect slightly differs from our main result due to the fact that matching is performed within industry only now. Although matched neighbours might be more 'distant' regarding other covariate values, we can ensure that each treated firm is allocated to a control observation with the same technology intensity. The covariate balances remain very good. 

%-----------------------TABLE 7
\begin{table}[h]
  \centering
   \caption{ATE by Technology Intensity of Industry}
   \label{tab:TECH}
\begin{threeparttable}
 
\begin{tabular}{lcccc}
 \hline
 \hline
 & & Medium & Medium &  \\ 
 & Low-Tech & Low-Tech & High-Tech & High-Tech \\ 
 & Industry & Industry & Industry & Industry \\ 
 & (1) & (2) & (3) & (4) \\
 \hline
 &  &  &  &  \\
FDI2016 & 0.160*** & 0.086*** & 0.172*** & 0.180*** \\
	      & (0.020) & (0.028) & (0.019) & (0.054) \\
	      &  &  &  &  \\
 Observations & 4,194 & 1,685 & 3,539 & 1,905 \\ 
	\hline
	\hline
	\multicolumn{5}{c}{\footnotesize{Standard errors in parentheses. *** p$<$0.01, ** p$<$0.05, * p$<$0.1. }}
\end{tabular}

\end{threeparttable}
\end{table}
%--------------------------------------


\section{Analysis by Type of FDI}


To further test the robustness of our results, we continue our analysis by looking at potential heterogeneity of the treatment effect across types of FDI. Recall the three types of FDI from table 1 as Exports oriented, Technology intensive and Domestic market seeking, with the the latter being most common and the first being least common.
We test the possibility that one specific type of investment single handedly drives our previous results.  It is possible, that for example only exports-oriented increases factor productivity while the other two types have little or no impact. 

We estimate an augmented IPW model with multi-valued treatment effects. The matching covariates are the same as in the previous model and the regression adjustment specification is the same as that for propensity score estimation. The covariate balance is good, but the variance ratio of employment (unlogged) is a borderline case. \footnote{Unfortunately an AIPW model with any interactions did not converge, so it was impossible to try and improve the balance this way.}%footnote sounds a bit like we failed in our analysis

We further estimate an IPW model to check if it yields similar estimates without regression adjustment. The covariate balance in this model is practically the same. Finally, we specify a set of AIPW models where we restrict the sample to one type of treatment each. This allows for the IIA assumption to be relaxed which is required for the mulitnominal logit models. The separate models have worse covariate balance than the first two but are borderline acceptable. The overlap assumption is satisfied for all treatment levels as can be seen in graph \ref{fig:psbytype}. 


In Table \ref{tab:bytype} the results from the type-wise analysis are shown. In the AIPW Multinomial specification the ATE of different types of FDI are within half a percent of of each other. This suggests, that all types of FDI increase factor productivity by essentially the same margin. The estimated effect size is close to the one estimated for FDI in table 2. In the IPW specification the differences slightly larger but still within 5\% of a standard deviation of each other.  The separate logit models also yield essentially the same effect sizes as the multinomial specification. Since the AIPW estimator is doubly robust (assuming correctly specified regression adjustment models) models (1) and (3) provide us with strong evidence of homogenous effects.


%-------------------FIGURE 2
\begin{figure}[h]\centering
\caption{Propensity Score by Treatment Level}
\includegraphics[height=11cm]{mlog_overl_ppb.png}\\[0.5cm] 
\label{fig:psbytype}
\end{figure}
%------------------------
% double headline should be removed


%-----------------------------TABLE 8
\begin{table}[htbp]
	\centering
	\caption{ATE by Type of FDI}
	\label{tab:bytype}
\begin{threeparttable}

\begin{tabular}{lccccc} 
		\hline
		\hline
 	& (1) & (2) & (3) & (4) & (5) \\
	& AIPW & IPW  & AIPW  & AIPW & AIPW \\ 
	& Mlogit & Mlogit &Logit &Logit &Logit\\
		\hline
 			&  &  &  &  &   \\
Exports-oriented FDI 	& 0.144*** &   0.157*** & 0.140*** &  &  \\
 						& (0.006) &   (0.032) & (0.007) &  &\\
Technology intensive FDI & 0.139***   & 0.112*** &  & 0.139*** &   \\
 						 & (0.005)  & (0.018) &  &  (0.005)&  \\
Domestic market seeking FDI & 0.143*** &   0.134*** &  &  &0.143*** \\
 							& (0.004)   & (0.011) &  &  & (0.004)  \\
PO Means 		&   -0.057*** &   -0.068*** &-0.012  &-0.025**  & -0.017    \\
 				&   (0.009) &   (0.010) &  (0.011)&(0.011)  & (0.011) \\
Observations 	& 11,323  & 11,323 &  7,803  & 8,418 & 8,828  \\ 
		\hline
		\hline
	\multicolumn{6}{c}{\footnotesize{Standard errors in parentheses. *** p$<$0.01, ** p$<$0.05, * p$<$0.1. }}
\end{tabular}

\end{threeparttable}
\end{table}
%------------------------------------------------


\section{Discussion/Conclusion}

%--------------how to cite
\comment{
For citation: \\
you have to add your reference firstly in bibCG. After having done so you can always include the reference in the actual file as follows: \\
 \citet{aitken99}\\
\citep[p.~35]{CaliendoHujerThomsen2008}	}
%--------------

Thoughts on what we could write for discussion/limits of our study: 
\begin{enumerate}
\item Do not know much about the context of the treatment (so cannot really rule out anticipation-effects?)
\item Would have been interesting to extend the study to several years after the treatment. Do effects persist? Do they vanish? 
\item Might depend on firm size (see \citet{aitken99}): find positive within-plant effects and spillover effects on TFP for small firms only (less than 50 employees)
\item Do not measure spillovers on plants that have not received FDI
\item Do not have sector-specific data $\rightarrow$ TECH variable has only 4 categories; e.g. in order to measure spillover effects from other firms in sector this would be necessary (i.e. if a foreign firm is more innovative)
\end{enumerate}

%maybe in conclusion/ limits of paper
One shortcoming of our analysis is the missing capability to explain the mechanisms behind the observed treatment effects. We do not distinguish between within-firm effects and spillover effects from firms receiving FDI to other firms. That is, the literature on spillover effects typically suggests spillovers within sectors rather than across sectors (SOURCE) and higher spillover effects on nearby firms or firms that are trading factor inputs with foreign-owned companies (SOURCE). Due to the limited information on a firm's sector and location, we are unable to measure spillover effects. 

\newpage

%-------------------------------------------------------------
% References
%-------------------------------------------------------------
\addcontentsline{toc}{section}{References}	%Adds references to table of contents
\bibliography{bibCG} 
\newpage


%-------------------------------------------------------------
% Appendix
%-------------------------------------------------------------
\pagenumbering{roman}
\sectionnumbering{Roman}
\setcounter{page}{3} %May have to adjust this if we leave out list of tables or add something else

\appendix
\section{Appendix}

%---------------------------FDI BY TECH (ROBUSTNESS)
\subsection{Treatment by Technology Intensity}
\label{app:tech}
\begin{table}[htbp]
	\centering
\begin{threeparttable}

\begin{tabular}{lcccccc} 
\hline
\hline
 & \multicolumn{6}{c}{FDI in 2016} \\
Technology intensity & \multicolumn{3}{c}{No} & \multicolumn{3}{c}{Yes} \\
of industry &No.&Col \% &Cum \% &No.&Col \% &Cum \% \\
\hline
	&  &  &  &  &  &   \\
Low-tech &1869&44.6&27.2&2325&55.4&52.1 \\
Medium low-tech &904&53.6&40.4&781&46.4&69.6 \\
Medium high-tech &2432&68.7&75.8&1107&31.3&94.5 \\
High-tech &1658&87.0&100.0&247&13.0&100.0 \\
\textbf{Total}&\textbf{6863}&\textbf{60.6}&&\textbf{4460}&\textbf{39.4}& \\
\hline
\hline
\end{tabular}

\end{threeparttable}
\end{table}
%-----------------------------------

\end{document}