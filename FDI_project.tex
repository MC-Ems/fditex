% -------------------------------------------------------------
%		Preamble
%--------------------------------------------------------------
% !TeX spellcheck = uk_UK
%Document Settings
\documentclass[a4paper,11pt]{scrartcl}
\usepackage[T1]{fontenc}
\usepackage{setspace}
\onehalfspacing
\DeclareOldFontCommand{\bf}{\normalfont\bfseries}{\mathbf} % Includes an old command that is part of some package in here (just to make sure erverything runs)

% Page numbers
\newcommand{\sectionnumbering}[1]{%
  \setcounter{section}{0}%
   \renewcommand{\thesection}{\csname #1\endcsname{section}}}

% Language settings
\usepackage[utf8]{inputenc}
\usepackage[english]{babel}

% Maths settings
\usepackage{amsmath}
\newcommand\numberthis{\addtocounter{equation}{1}\tag{\theequation}}
\usepackage{amssymb}
\usepackage{siunitx}

% Graphics packages
\usepackage{graphicx}
\graphicspath{{Graphics/}}
\usepackage{adjustbox}

% Table packages
\usepackage{pdflscape} %To create landscape environments
\usepackage{booktabs,caption}
\usepackage{threeparttable} %For nice tables
\usepackage{csvsimple} %To import excel tables 
\usepackage{import}

% Bilbiography settings
\usepackage{natbib}
\bibliographystyle{apalike}


% -------------------------------------------------------------
%		Title Page
%--------------------------------------------------------------
\begin{document}

	\begin{titlepage}
		\newcommand{\HRule}{\rule{\linewidth}{0.5mm}}
		
%% Logo
	\vfill\vfill
	\includegraphics[height=1.5cm]{UoN_Logo}\\[1cm] 


	\center			
%% Heading
	\textsc{\LARGE University of Nottingham}\\[1.5cm] 
	\textsc{\Large Applied Microeconometrics}\\[0.5cm] 	
	\textsc{\large Group Project A}\\[0.5cm] 
	
%% Title
	\HRule\\[0.4cm]
	{\huge\bfseries Insert Title}\\[0.4cm] 
	\HRule\\[0.4cm]
	
%% Date
	{\large\ Spring Term 2020} 	
	\vfill\vfill\vfill 		
	
%% Author(s) and Supervisor
\begin{flushleft}
			\large
			\textit{Supervisor}\\
			Professor Sourafel \textsc{Girma} 
			\vfill\vfill 
			\textit{Authors}\\
			Nelly  \textsc{Lehn} (20214338)\\
			Yonesse \textsc{Paris} (20115536)\\
			Thea  \textsc{Zoellner} (20216019)\\
			Georg  \textsc{Schneider} (20214032)\\
			Emilie \textsc{Bechtold} (20214031)
		\end{flushleft}
	\vfill 
	
\end{titlepage}


% -------------------------------------------------------------
%		Contents
%--------------------------------------------------------------
\pagenumbering{roman}
\sectionnumbering{Roman}
\tableofcontents

\newpage

\listoftables
\newpage

%-------------------------------------------------------------
% Main Body
%-------------------------------------------------------------
\pagenumbering{arabic}
\sectionnumbering{arabic}

\section{Introduction}



\section{Theoretical Background/Literature Review}

\subsection{FDI}

\subsection{PSM}
Since (I guess) we will be focussing on ATE rather than ATT, we need to satisfy the following two assumptions: 

\begin{enumerate}
\item Assumption: \textbf{Unconfoundedness (CIA)} \\
"\textit{[G]iven a set of observable covariates X which are not affected by treatment, potential outcomes are independent of treatment assignment}"   \citep[p.~35]{CaliendoHujerThomsen2008}	 

\item Assumption: \textbf{Overlap} \\
"\textit{persons with the same X values have a positive probability of being both participants and nonparticipants}" \citet [p.~35]{Caliendo08}

\end{enumerate}
--> if Assumption 1 holds, all biases due to observable components can be removed by conditioning on the propensity score (Imbens, 2004).

\subsubsection*{Binary Treatment}
Difference between logit and probit lies in the link function. Logit assumes a log-distribution of residuals, probit assumes a normal distribution. Heteroskedastic probit models can account for non-constant error variances --> Check for heteroskedasticity?

\subsubsection*{Multiple Treatments}
The multinomial probit model is the preferable option compared to logit. Alternatively, just run several binary ones (more complicated but also more robust to errors).

\subsubsection*{Variable selection}
\begin{itemize}
\item outcome variable must be independent of treatment conditional on the pscore (CIA)
\item Only variables that influence simultaneously the participation decision and the outcome variable should be included (based on theory and empirical findings)
\item variables should either be fixed over time or measured before participation (include only variables unaffeted by participation)
\item choice of variables should be based on economic theory and previous empirical findings
\end{itemize}

\subsubsection*{Tests for variable selection}
Strategies for the selection of variables to be used in estimating the propensity score:


Our analysis is based on observational firm-level data. The dataset comprises 11,323 firms, of which 4,460 received FDI in 2016. The FDIs are categorized into three different types: Exports-oriented, technology intensive and domestic market seeking FDI. The outcome variable TFP was measured in 2017. The baseline variables were measured in 2015 (one year prior to receiving FDI) and comprise information on:


\begin{itemize}
\item Ownership (listed company, subsidiary, independent or state owned)
\item Technology intensity (low, medium low, medium high or high-tech industries) 
\item Access to a port
\item Wages (as log variable)
\item Total Factor Productivity (TFP)
\item Firm size (measured in number of employees, log variable)
\item Debt (as log variable)
\item Export intensity
\item Whether the firm has invested in Research and Design
\end{itemize}



\section{Data and Descriptive Analysis}
\begin{table}

  \centering
   \caption{Frequency of FDI Types} 

\begin{tabular}{lcc} \hline
 FDI type&Abs. Freq.&Rel. Freq. \\
\hline
No FDI&6,863 & 61\% \\
Exports-oriented FDI&940 & 8\%\\
Technology intensive FDI & 1,555 & 14\% \\
Domestic market seeking FDI & 1,965 & 17\% \\
Total & 11,323 & 100\% \\
\end{tabular}
\end{table}


\section{Empirical Specification}

\subsection{Effect of FDI on TFP}

\begin{table}[htbp]\centering
\caption{Impact of FDI on TFP-Standardized}
\tiny
\begin{tabular}{lcccccccccc} \hline
 & NN1 & NN1 & NN5 & NN5 & IPW & IPW & IPW & IPW & AIPW & AIPW \\
VARIABLES & ATE & ATT & ATE& ATT & ATE & POmean & ATET & POmean & ATE & POmean \\ \hline
 &  &  &  &  &  &  &  &  &  &  \\
r1vs0.FDI2016 & 0.125*** & 0.147*** & 0.119*** & 0.133*** & 0.119*** &  & 0.179*** &  & 0.142*** &  \\
 & (0.019) & (0.020) & (0.013) & (0.011) & (0.006) &  & (0.006) &  & (0.003) &  \\
0.FDI2016 &  &  &  &  &  & -0.071*** &  & -0.199*** &  & -0.057*** \\
 &  &  &  &  &  & (0.010) &  & (0.016) &  & (0.009) \\
 &  &  &  &  &  &  &  &  &  &  \\
 Observations & 11,323 & 11,323 & 11,321 & 11,321 & 11,323 & 11,323 & 11,323 & 11,323 & 11,323 & 11,323 \\ \hline
\multicolumn{11}{c}{ Standard errors in parentheses} \\
\multicolumn{11}{c}{ *** p$<$0.01, ** p$<$0.05, * p$<$0.1} \\
\end{tabular}
\end{table}
 

\subsection{Robustness of Results}

Table 3 reports several alternative model specifications that confirm the robustness of our main results. The positive and significant effect of foreign investment on total factor productivity persists. In column (1), we add interaction terms of the dummy variables with the continuous regressors to our set of covariates. The average treatment effect of FDI on productivity slightly increases by 0.027 standard deviations. The covariate balance does not improve with the inclusion of interaction terms, suggesting that interactions do not increase the quality of matching.\footnote{The same holds true when interacting only dummy variables, only continuous variables or all variables.} One possible source of bias are the outliers of employment. While most of the firms' number of employees is concentrated around a mean of 7,111, we are concerned about two observations with extreme values: one firm with over eight million employees, and another one that apparently hired more than 4 million people in 2015.\footnote{Given the limited information our dataset contains, we cannot be sure in which unit employment is measured. We therefore suppose that the common definition of employment being the number of hired employees holds.}  To check whether these outliers influence our main findings, we exclude the two extreme observations. The results reported in column (2) show no significant change in the average treatment effect. %Note% should do F-test for equality of coefficients?%
In our main specification, we have further assumed that a nearby port does not influence total factor productivity. Column (3) confirms that this is indeed the case, as the results do not change with the inclusion of the dummy variable port in our set of covariates. 

 \begin{table}

  \centering
   \caption{Robustness of Results\textsuperscript{a}}

\begin{tabular}{lcccc} \hline
 & & & & Effect \\
 & Including & Excluding & Including & on the\\
 & Interactions\textsuperscript{b} & Outliers\textsuperscript{c} & Port\textsuperscript{d} & Treated \\
 & (1) & (2) & (3) & (4) \\ \hline
 &  &  &  &  \\
ATE & 0.152*** & 0.127*** & 0.125*** &  \\
 & (0.016) & (0.015) & (0.019) &  \\
 &  &  &  &  \\
 ATT &  &  &  & 0.127*** \\
 &   &  &  & (0.017) \\
 &  &  &  &  \\
 Observations & 11,323 & 11,321 & 11,323 & 11,323 \\ \hline
\multicolumn{5}{c}{ Standard errors in parentheses} \\
\multicolumn{5}{c}{ *** p$<$0.01, ** p$<$0.05, * p$<$0.1} \\
\end{tabular}
   \begin{tablenotes}
      \tiny
	\item\textsuperscript{a} All specifications use the Propensity Score Matching method with one nearest neighbor and replacement. 	Covariates include, unless otherwise specified: Ownership, Technology Intensity, Research\&Development, logarithm of Wages, Total Factor Productivity, Employment and Debts, all measured in 2015. 
	\item\textsuperscript{b} Interacting dummy variables (Ownership, Technology Intensity, Research \& Development) with continuous variables (Logarithm of wages, Total Factor Productivity, Employment and Debts).  
	\item\textsuperscript{c} Two observations with extreme values of Employment 2015 above four million are excluded. 
	\item\textsuperscript{d} Adding a dummy variable indicating whether a port lies within 500km of the firm to the set of covariates.
	\end{tablenotes}
\end{table}

Finally, column (4) reports the average treatment effect on the treated (ATT) of the nearest neighbor matching with one neighbor and replacement. Because we assume that treatment is not exogenous, we might find stronger effects of treatment on the treated. This would be the case if those firms receiving treatment are also the ones benefiting more from it. However, our estimate in column (4) reports an ATT that is very similar to the average treatment effect for the hypothetical case that all firms have received FDI. This suggests that the propensity score matching performed very well in randomizing treatment and control groups. 

%maybe in conclusion/ limits of paper
One shortcoming of our analysis is the missing capability of explaining the mechanisms behind the observed treatment effects. We do not distinguish between within-firm effects and spillover effects from firms receiving FDI on other firms. Due to the limited information on a firm's sector and location, we are unable to measure spillover effects. That is, the literature on spillover effects typically suggests spillovers within sectors rather than across sectors (SOURCE) and higher spillover effects on nearby firms or firms that are trading factor inputs with foreign-owned companies (SOURCE). 




\section{Analysis by Type}


To further test the robustness of our results  we continue our analysis by looking at potential heterogeneity of the treatment effect across types of FDI. Doing so we can test the possibility that one specific type of Investment drives our previous results. From the data we can distinguish between three different types of FDI: (i) exports-oriented FDI, (ii) technology intensive FDI and (iii) domestic market seeking FDI. Table 5 shows their absolute and relative frequencies. It is possible, that for example only exports-oriented increased factor productivity while the other two types had little or no impact.





To test for this possibility we estimate IPW and augmented IPW models with multi-valued treatment effects. The matching covariates are the same as in the previous model and the regression adjustment model is the same as that for propensity score estimation. The covariate balance is good in both models, 


In table 6 results from IPW and AIPW regressions are shown, where the treatment model is a Multinomial Logit with FDI types as unordered outcomes. The results show that all three types of FDI have positive and significant effects, so all types of FDI have a positive and significant effect on TFP. The effect sizes of the AIPW in table six are very close to those from the single valued treatment in table 4. Relative to the potential outcome mean of 3.5, the different types of FDI increase Factor productivity by 14\% of a standard deviation at at the 99\% level of significance.	The potential outcomes mean for receiving no FDI is 5.7\% below the sample mean. 

The Inverse Probability Weighting Model gives us somewhat bigger differences in effect sizes between the types. The difference between the bigger effect of Exports oriented FDI and the smaller effect of Technology intensive FDI amounts to 4\% of a standard deviation. Including interaction terms between continuous and categorical regressors 

With the AIPW being a doubly robust estimators and rather small differences in the IPW estimator we take these results to suggest that all types of FDI have similar positive impacts on factor productivity. 


\begin{table}[htbp]\centering
\caption{ATE by Type of FDI}
\begin{tabular}{lccccc} \hline
 & (1) & (2) & (3) & (4) & (5) \\
VARIABLES & AIPW & IPW  & AIPW  & AIPW & AIPW \\ 
&Mlogit&Mlogit&Logit&Logit&Logit\\
\hline
 &  &  &  &  &   \\
Exports-oriented FDI & 0.144*** &   0.157*** & 0.140*** &  &  \\
 & (0.006) &   (0.032) & (0.007) &  &\\
Technology intensive FDI & 0.139***   & 0.112*** &  & 0.139*** &   \\
 & (0.005)  & (0.018) &  &  (0.005)&  \\
Domestic market seeking FDI & 0.143*** &   0.134*** &  &  &0.143*** \\
 & (0.004)   & (0.011) &  &  & (0.004)  \\
PO Means &   -0.057*** &   -0.068*** &-0.012  &-0.025**  & -0.017    \\
 &   (0.009) &   (0.010) &  (0.011)&(0.011)  & (0.011) \\
 Observations & 11,323  & 11,323 &  7,803  & 8,418 & 8,828  \\ \hline
\multicolumn{6}{c}{ Robust standard errors in parentheses} \\
\multicolumn{6}{c}{ *** p$<$0.01, ** p$<$0.05, * p$<$0.1} \\
\end{tabular}
\end{table}

To account for the possibility that the choice of the FDI type does not satisfy the IIA assumption we further estimate separate logit models for the two estimator types. The results, reported in table 7, are very similar to those obtained from a multinomial specification, suggesting that the IIA assumption holds.  





\section{Discussion/Conclusion}
For citation: \\
you have to add your reference firstly in bibCG. After having done so you can always include the reference in the actual file as follows: \\
 \citet{biddle1990sleep}\\
\citep[p.~35]{CaliendoHujerThomsen2008}	 \\


Thoughts on what we could write for discussion/limits of our study: 
\begin{enumerate}
\item Do not know much about the context of the treatment (so cannot really rule out anticipation-effects?)
\item Would have been interesting to extend the study to several years after the treatment. Do effects persist? Do they vanish? 
\item Might depend on firm size (see Aitken \& Harrison 1999 $\rightarrow$ will include citation): find positive within-plant effects and spillover effects on TFP for small firms only (less than 50 employees)
\item Do not measure spillovers on plants that have not received FDI
\item Do not have sector-specific data $\rightarrow$ TECH variable has only 4 categories; e.g. in order to measure spillover effects from other firms in sector this would be necessary (i.e. if a foreign firm is more innovative)
\end{enumerate}


\newpage

%-------------------------------------------------------------
% References
%-------------------------------------------------------------
\addcontentsline{toc}{section}{References}	%Adds references to table of contents
\bibliography{bibCG} 
\newpage


%-------------------------------------------------------------
% Appendix
%-------------------------------------------------------------

\section*{Appendix}
\pagenumbering{roman}
\sectionnumbering{Roman}
\setcounter{page}{3} %May have to adjust this if we leave out list of tables or add something else

\end{document}