% -------------------------------------------------------------
%		Preamble
%--------------------------------------------------------------
% !TeX spellcheck = uk_UK
%Document Settings
\documentclass[a4paper,11pt]{scrartcl}
\usepackage[T1]{fontenc}
\usepackage{setspace}
\onehalfspacing
\DeclareOldFontCommand{\bf}{\normalfont\bfseries}{\mathbf} % Includes an old command that is part of some package in here (just to make sure erverything runs)

% Page numbers
\newcommand{\sectionnumbering}[1]{%
  \setcounter{section}{0}%
   \renewcommand{\thesection}{\csname #1\endcsname{section}}}

% Language settings
\usepackage[utf8]{inputenc}
\usepackage[english]{babel}

% Maths settings
\usepackage{amsmath}
\newcommand\numberthis{\addtocounter{equation}{1}\tag{\theequation}}
\usepackage{amssymb}
\usepackage{siunitx}

% Graphics packages
\usepackage{graphicx}
\graphicspath{{Graphics/}}
\usepackage{adjustbox}

% Table packages
\usepackage{pdflscape} %To create landscape environments
\usepackage{booktabs,caption}
\usepackage{threeparttable} %For nice tables
\usepackage{csvsimple} %To import excel tables 
\usepackage{import}

% Bilbiography settings
\usepackage{natbib}
\bibliographystyle{apalike}


% -------------------------------------------------------------
%		Title Page
%--------------------------------------------------------------
\begin{document}

	\begin{titlepage}
		\newcommand{\HRule}{\rule{\linewidth}{0.5mm}}
		
%% Logo
	\vfill\vfill
	\includegraphics[height=1.5cm]{UoN_Logo}\\[1cm] 


	\center			
%% Heading
	\textsc{\LARGE University of Nottingham}\\[1.5cm] 
	\textsc{\Large Applied Microeconometrics}\\[0.5cm] 	
	\textsc{\large Group Project A}\\[0.5cm] 
	
%% Title
	\HRule\\[0.4cm]
	{\huge\bfseries Insert Title}\\[0.4cm] 
	\HRule\\[0.4cm]
	
%% Date
	{\large\ Spring Term 2020} 	
	\vfill\vfill\vfill 		
	
%% Author(s) and Supervisor
\begin{flushleft}
			\large
			\textit{Supervisor}\\
			Professor Sourafel \textsc{Girma} 
			\vfill\vfill 
			\textit{Authors}\\
			Nelly  \textsc{Lehn} (20214338)\\
			Yonesse \textsc{Paris} (20115536)\\
			Thea  \textsc{Zoellner} (20216019)\\
			Georg  \textsc{Schneider} (20214032)\\
			Emilie \textsc{Bechtold} (20214031)
		\end{flushleft}
	\vfill 
	
\end{titlepage}


% -------------------------------------------------------------
%		Contents
%--------------------------------------------------------------
\pagenumbering{roman}
\sectionnumbering{Roman}
\tableofcontents

\newpage

\listoftables
\newpage

%-------------------------------------------------------------
% Main Body
%-------------------------------------------------------------
\pagenumbering{arabic}
\sectionnumbering{arabic}

\section{Introduction}



\section{Theoretical Background/Literature Review}

\subsection{FDI}

\subsection{PSM}
Since (I guess) we will be focussing on ATE rather than ATT, we need to satisfy the following two assumptions: 

\begin{enumerate}
\item Assumption: \textbf{Unconfoundedness (CIA)} \\
"\textit{[G]iven a set of observable covariates X which are not affected by treatment, potential outcomes are independent of treatment assignment}"   \citep[p.~35]{CaliendoHujerThomsen2008}	 

\item Assumption: \textbf{Overlap} \\
"\textit{persons with the same X values have a positive probability of being both participants and nonparticipants}" \citet [p.~35]{Caliendo08}

\end{enumerate}
--> if Assumption 1 holds, all biases due to observable components can be removed by conditioning on the propensity score (Imbens, 2004).

\subsubsection*{Binary Treatment}
Difference between logit and probit lies in the link function. Logit assumes a log-distribution of residuals, probit assumes a normal distribution. Heteroskedastic probit models can account for non-constant error variances --> Check for heteroskedasticity?

\subsubsection*{Multiple Treatments}
The multinomial probit model is the preferable option compared to logit. Alternatively, just run several binary ones (more complicated but also more robust to errors).

\subsubsection*{Variable selection}
\begin{itemize}
\item outcome variable must be independent of treatment conditional on the pscore (CIA)
\item Only variables that influence simultaneously the participation decision and the outcome variable should be included (based on theory and empirical findings)
\item variables should either be fixed over time or measured before participation (include only variables unaffeted by participation)
\item choice of variables should be based on economic theory and previous empirical findings
\end{itemize}

\subsubsection*{Tests for variable selection}
Strategies for the selection of variables to be used in estimating the propensity score:



\section{Data and Descriptive Analysis}
Our analysis is based on observational firm-level data. The dataset comprises 11,323 firms, of which 4,460 received FDI in 2016. The FDIs are categorized into three different types: Exports-oriented, technology intensive and domestic market seeking FDI. The outcome variable TFP was measured in 2017. The baseline variables were measured in 2015 (one year prior to receiving FDI) and comprise information on:

\begin{itemize}
\item Ownership (listed company, subsidiary, independent or state owned)
\item Technology intensity (low, medium low, medium high or high-tech industries) 
\item Access to a port
\item Wages (as log variable)
\item Total Factor Productivity (TFP)
\item Firm size (measured in number of employees, log variable)
\item Debt (as log variable)
\item Export intensity
\item Whether the firm has invested in Research and Design
\end{itemize}





\section{Empirical Specification}

\subsection{Effect of FDI on TFP}

\begin{table}[htbp]\centering
\caption{Impact of FDI on TFP-Standardized}
\tiny
\begin{tabular}{lcccccccccc} \hline
 & NN1 & NN1 & NN5 & NN5 & IPW & IPW & IPW & IPW & AIWP & AIWP \\
VARIABLES & ATE & ATT & ATE& ATT & ATE & POmean & ATET & POmean & ATE & POmean \\ \hline
 &  &  &  &  &  &  &  &  &  &  \\
r1vs0.FDI2016 & 0.125*** & 0.147*** & 0.119*** & 0.133*** & 0.119*** &  & 0.179*** &  & 0.142*** &  \\
 & (0.019) & (0.020) & (0.013) & (0.011) & (0.006) &  & (0.006) &  & (0.003) &  \\
0.FDI2016 &  &  &  &  &  & -0.071*** &  & -0.199*** &  & -0.057*** \\
 &  &  &  &  &  & (0.010) &  & (0.016) &  & (0.009) \\
 &  &  &  &  &  &  &  &  &  &  \\
 Observations & 11,323 & 11,323 & 11,321 & 11,321 & 11,323 & 11,323 & 11,323 & 11,323 & 11,323 & 11,323 \\ \hline
\multicolumn{11}{c}{ Standard errors in parentheses} \\
\multicolumn{11}{c}{ *** p$<$0.01, ** p$<$0.05, * p$<$0.1} \\
\end{tabular}
\end{table}


\begin{table}[htbp]\centering
\caption{Impact of FDI on TFP}
\tiny
\begin{tabular}{lccccccc} \hline
 & NN1 & NN1 & NN5 & NN5 & IWP & IPW & AIWP \\
VARIABLES & ATE & ATT & ATE & ATT & ATE & ATT & ATE\\ \hline
 &  &  &  &  &  &  &  \\
r1vs0.FDI2016 & 0.257*** & 0.302*** & 0.246*** & 0.273*** & 0.245***   & 0.367*** & 0.292***  \\
 & (0.038) & (0.040) & (0.028) & (0.022) & (0.013)  & (0.013) & (0.006) \\
0.FDI2016 & & & & & &\\
P0  Means &  &  &  &    & 3.510***  & 3.247***& 3.540*** \\
 &  &  &  &  & (0.020)  & (0.033) & (0.020) \\
 &  &  &  &  &  &  &    \\
Observations & 11,323 & 11,321 & 11,321 & 11,323 & 11,323 & 11,323 & 11,323\\  \hline
\multicolumn{8}{c}{ Standard errors in parentheses} \\
\multicolumn{8}{c}{ *** p$<$0.01, ** p$<$0.05, * p$<$0.1} \\
\end{tabular}
\end{table}


\section{Analysis by Type}


To further test the robustness of our results  we continue our analysis by looking at potential heterogeneity of the treatment effect across types of FDI. Doing so we can test the possibility that one specific type of Investment drives our previous results. From the data we can distinguish between three different types of FDI: (i) exports-oriented FDI, (ii) technology intensive FDI and (iii) domestic market seeking FDI. Table 5 shows their absolute and relative frequencies. It is possible, that for example only exports-oriented increased factor productivity. while the other two types had little or no impact.


\begin{table}[htbp]\centering
\caption{Types of FDI}
\begin{tabular}{lcc} \hline
 FDI type&No. \\
\hline
No FDI&6,863.0 \\
Exports-oriented FDI&940 \\
Technology intensive FDI&1,555 \\
 Domestic market seeking FDI&1,965 \\
Total&11,323.0 \\
\end{tabular}
\end{table}



\begin{table}[htbp]\centering
\caption{Multinomial Logits}
\begin{tabular}{lcc} \hline
 & (1) & (2)  \\
VARIABLES & AIPW & IPW \\ \hline
 &    &  \\
Exports oriented FDI &  0.141*** &   0.155***  \\
 & (0.006) &   (0.028)   \\
Technology intensive FDI &0.139*** &   0.114***   \\
 & (0.005) &   (0.016)   \\
Domestic market seeking FDI  &0.143***   & 0.123*** \\
 & (0.004)   & (0.010)  \\
PO Means & -0.057*** &   -0.071***    \\
 &  (0.009) &   (0.010) \\
 &  &      \\
 Observations & 11,323  & 11,323  \\ \hline
\multicolumn{3}{c}{ Robust standard errors in parentheses} \\
\multicolumn{3}{c}{ *** p$<$0.01, ** p$<$0.05, * p$<$0.1} \\
\end{tabular}
\end{table}

\begin{table}[htbp]\centering
\caption{Seperate Logits for FDI Type}
\begin{tabular}{lcccccc} \hline
 & Export   &Export  & Technology  &Technology  & Domestic  & Domestic  \\
VARIABLES & AIPW & IPW & AIPW & IPW & AIPW & IPW  \\ \hline
 &  &  &  &  &  &  \\
ATE  &0.136*** &   0.110*** &   0.139*** &   0.078*** &   0.143***   & 0.095***   \\
 & (0.007) &  (0.037)   & (0.005)   & (0.020) & (0.004) & (0.012)  \\
PO Mean &   -0.013 &   -0.014 &   -0.025** &   -0.028***   & -0.017 &   -0.022** \\
 &   (0.011) &   (0.011) &  (0.011)   & (0.011)   & (0.011)  & (0.011) \\
 &  &  &  &  &  & \\
 Observations & 7,803  & 7,803&  8,418 & 8,418  &  8,828  & 8,828 \\ \hline
\multicolumn{7}{c}{ Robust standard errors in parentheses} \\
\multicolumn{7}{c}{ *** p$<$0.01, ** p$<$0.05, * p$<$0.1} \\
\end{tabular}
\end{table}



\section{Discussion/Conclusion}
For citation: \\
you have to add your reference firstly in bibCG. After having done so you can always include the reference in the actual file as follows: \\
 \citet{biddle1990sleep}\\
\citep[p.~35]{CaliendoHujerThomsen2008}	 \\


Thoughts on what we could write for discussion/limits of our study: 
\begin{enumerate}
\item Do not know much about the context of the treatment (so cannot really rule out anticipation-effects?)
\item Would have been interesting to extend the study to several years after the treatment. Do effects persist? Do they vanish? 
\item Might depend on firm size (see Aitken \& Harrison 1999 $\rightarrow$ will include citation): find positive within-plant effects and spillover effects on TFP for small firms only (less than 50 employees)
\item Do not measure spillovers on plants that have not received FDI
\item Do not have sector-specific data $\rightarrow$ TECH variable has only 4 categories; e.g. in order to measure spillover effects from other firms in sector this would be necessary (i.e. if a foreign firm is more innovative)
\item Propensity Score matching might not be the best approach for given data as CIA could be violated
\end{enumerate}


\newpage

%-------------------------------------------------------------
% References
%-------------------------------------------------------------
\addcontentsline{toc}{section}{References}	%Adds references to table of contents
\bibliography{bibCG} 
\newpage


%-------------------------------------------------------------
% Appendix
%-------------------------------------------------------------

\section*{Appendix}
\pagenumbering{roman}
\sectionnumbering{Roman}
\setcounter{page}{3} %May have to adjust this if we leave out list of tables or add something else

\end{document}