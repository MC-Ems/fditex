% -------------------------------------------------------------
%		Preamble
%--------------------------------------------------------------
%Document Settings
\documentclass[a4paper,12pt]{scrartcl}
\usepackage[T1]{fontenc}
\usepackage{setspace}
\linespread{1.5}

% Page numbers
\newcommand{\sectionnumbering}[1]{%
  \setcounter{section}{0}%
   \renewcommand{\thesection}{\csname #1\endcsname{section}}}

% Language settings
\usepackage[utf8]{inputenc}
\usepackage[english]{babel}

% Maths settings
\usepackage{amsmath}
\newcommand\numberthis{\addtocounter{equation}{1}\tag{\theequation}}
\usepackage{amssymb}
\usepackage{siunitx}

% Graphics packages
\usepackage{graphicx}
\graphicspath{{Graphics/}}
\usepackage{adjustbox}
\usepackage{lscape}

% Bilbiography settings
\usepackage{natbib}
\bibliographystyle{apalike}
\setcitestyle{authoryear,open={(},close={)}}



% -------------------------------------------------------------
%		Title Page
%--------------------------------------------------------------
\begin{document}

	\begin{titlepage}
		\newcommand{\HRule}{\rule{\linewidth}{0.5mm}}
		
%% Logo
	\vfill\vfill
	\includegraphics[height=1.5cm]{UoN_Logo}\\[1cm] 


	\center			
%% Heading
	\textsc{\LARGE University of Nottingham}\\[1.5cm] 
	\textsc{\Large Applied Microeconometrics}\\[0.5cm] 	
	\textsc{\large Group Project A}\\[0.5cm] 
	
%% Title
	\HRule\\[0.4cm]
	{\huge\bfseries The effect of FDI on Total Factor Productivity and Wages}\\[0.4cm] 
	\HRule\\[0.4cm]
	
%% Date
	{\large\ Spring Term 2020} 	
	\vfill\vfill\vfill 		
	
%% Author(s) and Supervisor
\begin{flushleft}
			\large
			\textit{Supervisor}\\
			Professor Sourafel \textsc{Girma} 
			\vfill\vfill 
			\textit{Authors}\\
			Yonesse \textsc{Paris} (stud. n)\\
			Nelly  \textsc{Lehn} (20214338)\\
			Thea  \textsc{Zoellner} (stud. n)\\
			Georg  \textsc{Schneider} (stud. n)\\
			Emilie \textsc{Bechtold} (20214031)
		\end{flushleft}
	\vfill 
	
\end{titlepage}


% -------------------------------------------------------------
%		Contents
%--------------------------------------------------------------
\pagenumbering{roman}
\sectionnumbering{Roman}
\tableofcontents

\newpage

\listoftables
\newpage

%-------------------------------------------------------------
% Main Body
%-------------------------------------------------------------
\pagenumbering{arabic}
\sectionnumbering{arabic}



\section{Theoretical Background/Literature Review}

\subsection{FDI}

\subsection{PSM}
Since (I guess) we will be focussing on ATE rather than ATT, we need to satisfy the following two assumptions: 

\begin{enumerate}
\item Assumption: \textbf{Unconfoundedness (CIA)} \\
"\textit{[G]iven a set of observable covariates X which are not affected by treatment, potential outcomes are independent of treatment assignment}" (Caliendo \& Kopeinig, 2008: 35).

\item Assumption: \textbf{Overlap} \\
"\textit{persons with the same X values have a positive probability of being both participants and nonparticipants}"(Caliendo \& Kopeinig, 2008: 35).

\end{enumerate}
--> if Assumption 1 holds, all biases due to observable components can be removed by conditioning on the propensity score (Imbens, 2004).

\subsubsection*{Binary Treatment}
Difference between logit and probit lies in the link function. Logit assumes a log-distribution of residuals, probit assumes a normal distribution. Heteroskedastic probit models can account for non-constant error variances --> Check for heteroskedasticity?

\subsubsection*{Multiple Treatments}
The multinomial probit model is the preferable option compared to logit. Alternatively, just run several binary ones (more complicated but also more robust to errors).

\subsubsection*{Variable selection}
\begin{itemize}
\item outcome variable must be independent of treatment conditional on the pscore (CIA)
\item Only variables that influence simultaneously the participation decision and the outcome variable should be included (based on theory and empirical findings)
\item variables should either be fixed over time or measured before participation (include only variables unaffeted by participation)
\item choice of variables should be based on economic theory and previous empirical findings
\end{itemize}

\subsubsection*{Tests for variable selection}
Strategies for the selection of variables to be used in estimating the propensity score:



\section{Data and Descriptive Analysis}

\section{Empirical Specification}
\subsection{Econometric approach}
%-------------------------------------------------------------
%  COMMENTS: DELETE later
%---------------------------------------------------------
\textbf{Reminder of a thought we had}\\
We could drop all the state-owned enterprises, because wages are likely not to change just because the firm received foreign investment. \\\\
COMMENT: should unlog employment! $->$ gives much better t$-$differences in robustness check (see robust script last part)\\
%-------------------------------------------------------------
% Problem  of non-random treatment assignment + proposed Solution 
%-------------------------------------------------------------

1) Big model: including all variables does not seem to satisfy CIA???. Tech seems to predict treatment and control too well, especially for firms operating in the high-technology sector (TECH=4). High technology firms seem to have very low probability to receive treatment (see Graph). 
As treatment is  highly probable to be not randomly assigned we use propensity score matching, assuming CIA. 

In order to estimate the effect of FDI on TFP we firstly look at different model specifications as also model estimators. 
As described before including TECH leads to very bad overlap, for which we exclude latter. Further, excluding TECH leads to better balance, as Table X shows.   Graph XY visualizes that excluding TECH yields a better overlap.l
Specifying the effect of FDI on TFP for different types of TECH reduces the sample size drastically. The sample is unlikely to be representative. 
Furthermore, "good" matching within TECH types is difficult, especially as more control  than  treatment observations are available. 

Do we have to / Can we include a formula describing our estimation model? 
like in OLS y= ß0+Xß1+e

Matching with replacement minimizes propensity score distance between matched comparison units and treatment units => beneficial in terms of bias reduction -> following advice of Dehejia and Wahba (2002) of using with replacement if overlap is bad 
%-------------------------------------------------------------
% Results: Wages and TFP, NN1, NN5 (Table 1)
%-------------------------------------------------------------
\subsection{Main Results}
- COMMENT: delete table 1 and merge table 2 and 3 to 'table 1' but 'expanding' table vertically, not horizontally\\
- QUESTION: NN1 is equal to psmatch?--> so in the command nn1 does not have to be specified, unlike nn5\\
- QUESTION: is our caliper too big? see DW (2002): use caliper of 0.0001\\
- Delete Wages??\\

The main findings of this paper are displayed in table 1. It reports the average treatment effects (ATE) and the average treatment effects on the treated (ATT) for the variables of interest, total factor productivity (TFP) and wages. We find positive and significant treatment effects for both outcomes. Column (1) and (2) are the results of a one-to-one propensity score matching with replacement. Receiving FDI leads to an increase in wages of 0.139 (how to interpret ATE - in log form?) but this effect vanishes and becomes insignificant when looking at the treated sample only. This suggests that if some firms receive FDI, this does not have an effect on the treated themselves but on those firms that did not even receive FDI (ODD or interesting??).  Choosing total factor productivity as an outcome variable instead implies that a firm that attracted FDI experiences a significant improvement in total factor productivity of 0.287 (?) one year later and this effect increases to 0.312 when looking at treated firms only. Similar results are obtained from a propensity score matching with five nearest neighbors and a caliper of 0.05. As shown in columns (3) and (4), the effects on total factor productivity are fairly similar to those in the first two columns. Both the ATE and the ATT for wages increase when matching with five nearest neighbors, but the ATT remains insignificant. \\

COMMENT: include note under table that specifies which covariates were used in order to estimate the propensity score

\begin{table}[htbp]\centering
\caption{logWages and Total Factor Productivity}
\begin{tabular}{lcc} \hline
 &NN1 & NN1\\
VARIABLES & logWages2017 & TFP2017 \\ \hline
 &  &  \\
r1vs0.FDI2016 & 0.139** & 0.287*** \\
 & (0.067) & (0.040) \\
 &  &  \\
 Observations & 11,323 & 11,323 \\ \hline
\multicolumn{3}{c}{ Standard errors in parentheses} \\
\multicolumn{3}{c}{ *** p$<$0.01, ** p$<$0.05, * p$<$0.1} \\
\end{tabular}
\end{table}


\begin{table}[htbp]\centering
\caption{logWages 2017}
\begin{tabular}{lcccc} \hline
 & NN1 & NN1 & NN5 & NN5 \\
VARIABLES & ATE & ATET & ATE & ATET\\ \hline
 &  &  &  &  \\
r1vs0.FDI2016 & 0.139** & 0.037 & 0.187*** & 0.137 \\
 & (0.067) & (0.125) & (0.054) & (0.085) \\
 &  &  &  &  \\
 Observations & 11,323 & 11,323 & 11,317 & 11,317 \\ \hline
\multicolumn{5}{c}{ Standard errors in parentheses} \\
\multicolumn{5}{c}{ *** p$<$0.01, ** p$<$0.05, * p$<$0.1} \\
\end{tabular}
\end{table}

\begin{table}[htbp]\centering
\caption{TFP 2017}
\begin{tabular}{lcccc} \hline
 & NN1 & NN1 & NN5 & NN5 \\
VARIABLES & ATE & ATET & ATE & ATET \\ \hline
 &  &  &  &  \\
r1vs0.FDI2016 & 0.287*** & 0.312*** & 0.279*** & 0.318*** \\
 & (0.040) & (0.057) & (0.033) & (0.045) \\
 &  &  &  &  \\
 Observations & 11,323 & 11,323 & 11,317 & 11,317 \\ \hline
\multicolumn{5}{c}{ Standard errors in parentheses} \\
\multicolumn{5}{c}{ *** p$<$0.01, ** p$<$0.05, * p$<$0.1} \\
\end{tabular}
\end{table}

\newpage
%-------------------------------------------------------------
% Table 4: TFP 2017: IPW, AIPW
%-------------------------------------------------------------
Table 4 shows parameter estimates for \textit{FDI}  on  \textit{TFP} when using different  estimators for the treatment effects. First, we use an inverse-probability estimator. We estimate the effect of \textit{FDI}  on \textit{TFP} by using a logit model to predict the effect of \textit{FDI} as a function of \textit{port}, \textit{logwages 2015}, \textit{TFP2015}, \textit{logemployment2015}, \textit{debts2015} and \textit{RD2015}. The estimated ATET of \textit{FDI}  on \textit{TFP} is .308. Thus the average company in the treated population will increase its \textit{TFP} by 0.308  more than it would if no \textit{FDI}  had taken place. The ATE is slightly lower. The average company will increase its \textit{TFP} by 0.285 when it receives \textit{FDI}. Here the ATE is slightly lower than the ATET. It can be suspected that the treatment assignment mechanism was potentially non random, in which case the ATE should not be the major estimator. ????check theory. This could be the case in this data as the probability of getting an \textit{FDI}  differs depending on???

Second, we model the outcome as a linear function of before defined control variables. Again we use a logit model, where the covariates are also explanatory variables. \textit{FDI} increases \textit{TFP} on average by 0.306 from the average  \textit{TFP} 3.537 of firms which do not receive \textit{FDI}. All coefficients are significant to a 1$-$percent level. 

\begin{table}[htbp]\centering
\caption{Total Factor Productivity 2017}
\begin{tabular}{lcccc} \hline
 & IPW  & IPW & AIWP \\
VARIABLES & ATE  & ATET  & ATE & \\ \hline
 &  &  &  \\
r1vs0.FDI2016 & 0.285***  & 0.308*** &   0.306***   \\
 \bigskip
 & (0.029)  & (0.045)  & (0.010)   \\

 
0.FDI2016  &   &    &  \\

P0mean &3.537***  &   3.307*** & 3.537***  \\

 &   (0.026) & (0.053) & (0.020) \\
 &  &  &    \\
 Observations & 11,323 & 11,323 & 11,323  \\ \hline
\multicolumn{5}{c}{ Robust standard errors in parentheses} \\
\multicolumn{5}{c}{ *** p$<$0.01, ** p$<$0.05, * p$<$0.1} \\
\end{tabular}
\end{table}

\newpage
%-------------------------------------------------------------
% Discussion
%-------------------------------------------------------------
\subsection{Discussion}
Specify which estimator we prefer.\\
- look at covariate balance \\
==> choose "worst-best"estimator: Tell reasons why best estimator as also why Propensity Score Matching might not be the technique to use as CIA is not satisfied. 

%-------------------------------------------------------------
% Robustness
%-------------------------------------------------------------
\subsection{Robustness}
Look at Robustness checks for chosen "worst-best" estimator \\
-Model Specification: \\
-Interaction Term\\
-T-test : NN5 better than NN1, not possible for IPW, AIWP??






\section{FDI by type}

\section{Summary/ Conclusion}




\newpage

\bibliography{bibCG}


%-------------------------------------------------------------
% References
%-------------------------------------------------------------
%\printbibliography (Not quite sure how this works yet...)

%-------------------------------------------------------------
% Appendix
%-------------------------------------------------------------

\section*{Appendix}
\pagenumbering{roman}
\sectionnumbering{Roman}
\setcounter{page}{3} %%May have to adjust this if we leave out list of tables or add something else

\end{document}